
\section{\label{sec:successiveColBasisComputation}A Simple Improvement}

When the input matrix $\mathbf{F}$ has column dimension much larger
$n$ than the row dimension $m$, we can separate $\mathbf{F}=\left[\mathbf{F}_{1},\mathbf{F}_{2},\dots,\mathbf{F}_{n/m}\right]$
to $n/m$ blocks, each with dimension $m\times m$, assuming without
loss of generality $n$ is a multiple of $m$, and the columns are
arranged in increasing degrees. We then do a series of column basis
computations. First we compute a column basis $\mathbf{T}_{1}$ of
$\left[\mathbf{F}_{1},\mathbf{F}_{2}\right]$. Then compute a column
basis $\mathbf{T}_{2}$ of $\left[\mathbf{T}_{1},\mathbf{F}_{3}\right]$.
Repeating this process, at step $i$, we compute a column basis $\mathbf{T}_{i}$
of $\left[\mathbf{T}_{i-1},\mathbf{F}_{i+1}\right]$, until $i=n/m-1$,
when a column basis of $\mathbf{F}$ is computed. 
\begin{lem}
At step $i$, computing a column basis $\mathbf{T}_{i}$ of $\left[\mathbf{T}_{i-1},\mathbf{F}_{i+1}\right]$
can be done with a cost of $O^{\sim}\left(m^{\omega}(s_{i}+s_{i+1})/2\right)$
field operations, where $s_{i}=\left(\sum\cdeg\mathbf{F}_{i}\right)/m$.\end{lem}
\begin{proof}
From Lemma \ref{lem:colBasisdegreeBoundByRdegOfRightFactor}, the
column basis $\mathbf{T}_{i-1}$ of $\left[\mathbf{F}_{1},\dots,\mathbf{F}_{i}\right]$
has column degrees bounded by the largest column degrees of $\mathbf{F}_{i}$,
hence $\sum\cdeg\mathbf{T}_{i-1}\le\sum\cdeg\mathbf{F}_{i}$. The
lemma then follows by combining this with the result from Theorem
\ref{thm:columnBasisCost1} that a column basis $\mathbf{T}_{i}$
of $\left[\mathbf{T}_{i-1},\mathbf{F}_{i+1}\right]$ can be computed
with a cost of $O^{\sim}\left(m^{\omega}\bar{s}\right)$, where 
\[
\bar{s}=\left(\sum\cdeg\mathbf{T}_{i-1}+\sum\cdeg\mathbf{F}_{i+1}\right)/2m\le\left(s_{i}+s_{i+1}\right)/2.
\]
 \end{proof}
\begin{thm}
\label{thm:finalCollBasisCost}A column basis of $\mathbf{F}$ can
be computed with a cost of $O^{\sim}\left(m^{\omega}s\right)$, where
$s=\left(\sum\cdeg\mathbf{F}\right)/n$.\end{thm}
\begin{proof}
Summing up the cost of all the column basis computations, 
\begin{eqnarray*}
 &  & \sum_{i=1}^{n/m-1}O^{\sim}\left(m^{\omega}\left(s_{i}+s_{i+1}\right)/2\right)\\
 &  & ~~~~~~~~~~~~~\subset O^{\sim}\left(m^{\omega}\left(\sum_{i=1}^{n/m}s_{i}\right)/(n/m)\right)\\
 &  & ~~~~~~~~~~~~~=O^{\sim}\left(m^{\omega}s\right).
\end{eqnarray*}
\end{proof}
\begin{rem}
In this section, the computational efficiency is improved by reducing
the original problem to about $n/m$ subproblems whose column dimensions
are close to the row dimension $m$. This is done by successive column
basis computations. Note that we can also reduce the column dimension
by using successive order basis computations, and only do a column
basis computation at the very last step. The computational complexity
of using order basis computation to reduce the column dimension would
remain the same, but in practice it maybe more efficient since order
basis computations are simpler. \end{rem}

