
\section{Computing a Column Basis}

Once a right factor $\mathbf{G}$ of $\mathbf{F}$ has been computed,
we are in a position to determine a column basis $\mathbf{T}$ using
the equation $\mathbf{F}=\mathbf{T}\mathbf{G}$. In order to do so
efficiently, however, the degree of $\mathbf{T}$ cannot be too large.
We see that this is the case from the following lemma. 
\begin{lem}
\label{lem:colBasisdegreeBoundByRdegOfRightFactor} Let $\mathbf{F}$
and $\mathbf{G}$ be as before and $\vec{t}=-\rdeg_{-\vec{s}}\mathbf{G}$.
Then 
\begin{itemize}
\item[(i)] the column degrees of $\mathbf{T}$ are bounded by the corresponding
entries of $\vec{t}$; 
\item[(ii)] if $\vec{t}$ has $r$ entries and $\vec{s}^{~\prime}$ is the list
of the $r$ largest entries of $\vec{s}$, then $\vec{t}\le\vec{s}^{~\prime}$. 
\end{itemize}
\end{lem}
\begin{proof}
Since $\mathbf{G}$ is $-\vec{s}$-row reduced, and $\rdeg_{-\vec{s}}\mathbf{F}\le0$,
by Lemma \ref{lem:predictableDegree} $\rdeg_{-\vec{t}}\mathbf{T}\le0$,
or equivalently, $\mathbf{T}$ has column degrees bounded by $\vec{t}$.

Let $\mathbf{G}^{\prime}$ be the $-\vec{s}$-row Popov form of $\mathbf{G}$
and the square matrix $\mathbf{G}^{\prime\prime}$ consist of only
the columns of $\mathbf{G}^{\prime}$ that contains pivot entries,
and has the rows permuted so the pivots are in the diagonal. Let $\vec{s}^{~\prime\prime}$
be the list of the entries in $\vec{s}$ that correspond to the columns
of $\mathbf{G}^{\prime\prime}$ in $\mathbf{G}^{\prime}$. Note that
$\rdeg_{-\vec{s}^{~\prime\prime}}\mathbf{G}^{\prime\prime}=-\vec{t}^{~\prime\prime}$
is just a permutation of $-\vec{t}$ with the same entries. By the
definition of shifted row degree, $-\vec{t}^{~\prime\prime}$ is the
sum of $-\vec{s}^{~\prime\prime}$ and the list of the diagonal pivot
degrees, which are nonnegative. Therefore, $-\vec{t}^{~\prime\prime}\ge-\vec{s}^{~\prime\prime}$.
The result then follows as $\vec{t}$ is a permutation of $\vec{t}^{~\prime\prime}$
and $\vec{s}^{\ \prime}$ consists of the largest entries of $\vec{s}$. 
\end{proof}
Having determined a bound on the column degrees of $\mathbf{T}$,
we are now ready to compute $\mathbf{T}$. This is done again by computing
a kernel basis using an order basis computation as before. 
\begin{lem}
Let $\vec{t}^{*}=\left[0,\dots,0,\vec{t}\right]\in\mathbb{Z}^{m+r}$.
Then any $\left(\left[\mathbf{F}^{T},\mathbf{G}^{T}\right],-\vec{t}^{*}\right)$-kernel
basis has the form $\begin{bmatrix}V\\
\bar{\mathbf{T}}
\end{bmatrix}$, where $V\in\mathbb{K}^{m\times m}$ is a unimodular matrix and $\left(\bar{\mathbf{T}}V^{-1}\right)^{T}$
is a column basis of $\mathbf{F}$. \end{lem}
\begin{proof}
Note first that the matrix $\begin{bmatrix}-I~\\
~ \mathbf{T}^{T}
\end{bmatrix}$ is a kernel basis of $\left[\mathbf{F}^{T},\mathbf{G}^{T}\right]$
and is therefore unimodularly equivalent to any other kernel basis.
Hence any other kernel basis has the form $\begin{bmatrix}-I~\\
~\mathbf{T}^{T}
\end{bmatrix}U=\begin{bmatrix}V\\
\bar{\mathbf{T}}
\end{bmatrix}$, with $U$ and $V=-U$ unimodular. Thus $\mathbf{T}=\left(\bar{\mathbf{T}}V^{-1}\right)^{T}$.
Also note that the $-\vec{t}^{*}$ minimality forces the unimodular
matrix $V$ in any $\left(\left[\mathbf{F}^{T},\mathbf{G}^{T}\right],-\vec{t}^{*}\right)$-kernel
basis to be degree 0, the same degree as $I$. \end{proof}
\begin{exmp}
Let 
\[
\mathbf{F}=\left[\begin{array}{cccc}
x^{2} & x^{2} & x+x^{2} & 1+x^{2}\\
1+x+x^{2} & x^{2} & 1+x^{2} & 1+x^{2}
\end{array}\right]
\]
with 
\[
\mathbf{G}=\left[\begin{array}{cccc}
1 & 0 & 1 & 0\\
\noalign{\medskip}x & {x}^{2} & 0 & 1+{x}^{2}
\end{array}\right]
\]
being a minimal left kernel basis of a right kernel basis of $\mathbf{F}$.
In order to compute the column basis $\mathbf{T}$ satisfying $\mathbf{F}=\mathbf{T}\mathbf{G}$,
first we can determine $\cdeg\mathbf{T}\le\vec{t}=\left[2,0\right]$
from Lemma \ref{lem:colBasisdegreeBoundByRdegOfRightFactor}. Then
we can compute a $\left[0,0,-\vec{t}\right]$-minimal left kernel
basis of $\begin{bmatrix}\mathbf{F}\\
\mathbf{G}
\end{bmatrix}$. The matrix 
\[
\left[V,\bar{\mathbf{T}}\right]=\left[\begin{array}{cccc}
\noalign{\medskip}1 & 0 & x+{x}^{2} & 1\\
1 & 1 & 1+x & 0
\end{array}\right]
\]
is such a left kernel basis. A column basis can then be computed as
by 
\[
\mathbf{T}=V^{-1}\bar{\mathbf{T}}=\left[\begin{array}{cc}
x+x^{2} & 1\\
1+{x}^{2} & 1
\end{array}\right].
\]

\end{exmp}
In order to compute a $\left(\left[\mathbf{F}^{T},\mathbf{G}^{T}\right],-\vec{t}^{*}\right)$-kernel
basis, we can again use order basis computation as before, as we again
have an order basis that contains a $\left(\left[\mathbf{F}^{T},\mathbf{G}^{T}\right],-\vec{t}^{*}\right)$-kernel
basis.
\begin{lem}
Any $\left(\left[\mathbf{F}^{T},\mathbf{G}^{T}\right],\vec{s}+1,-\vec{t}^{*}\right)$
order basis contains a $\left(\left[\mathbf{F}^{T},\mathbf{G}^{T}\right],-\vec{t}^{*}\right)$-kernel
basis whose $-\vec{t}^{*}$-row degrees are bounded by 0. 
\end{lem}
\begin{proof}
As before, Lemma \ref{lem:orderbasisContainsKernelbasisGeneralized}
can be used here. We just need to show that a $\left(\left[\mathbf{F}^{T},\mathbf{G}^{T}\right],-\vec{t}^{*}\right)$-kernel
basis has $-\vec{t}^{*}$-row degrees no more than $0$. This follows
from the fact that $\rdeg_{-\vec{t}^{*}}\begin{bmatrix}I\\
\mathbf{T}^{T}
\end{bmatrix}\le0$. 
\end{proof}
In order to compute a $\left(\left[\mathbf{F}^{T},\mathbf{G}^{T}\right],-\vec{t}^{*}\right)$
kernel basis efficiently, we notice that we have the same type of
problem as in Section \ref{sub:kernelBasisViaOrderBasisByRows} and
hence we can again use Algorithm \ref{alg:minimalKernelBasisReverse}. 
\begin{lem}
\label{lem:costOfKernelBasisReversedForLeftFactor}A $\left(\left[\mathbf{F}^{T},\mathbf{G}^{T}\right],-\vec{t}^{*}\right)$-kernel
basis can be computed using \prettyref{alg:minimalKernelBasisReverse}
with a cost of $O^{\sim}\left(n^{\omega}s\right)$, where $s=\xi/n$
is the average column degree of $\mathbf{F}$ as before. \end{lem}
\begin{proof}
Just use the algorithm with input $\left(\left[\mathbf{F}^{T},\mathbf{G}^{T}\right],\vec{t}^{*},\xi\right)$.
We can verify the conditions on the input are satisfied.
\begin{itemize}
\item To see that $\sum\rdeg_{\vec{t}^{*}}\left[\mathbf{F}^{T},\mathbf{G}^{T}\right]\le\xi$,
note that from $\vec{t}=-\rdeg_{-\vec{s}}\mathbf{G}$ and \prettyref{lem:columnDegreesRowDegreesSymmetry} that 
$\cdeg_{\vec{t}}\mathbf{G}\le\vec{s}$, or equivalently, $\rdeg_{\vec{t}}\mathbf{G}^{T}\le\vec{s}$.
Since we also have $\rdeg\mathbf{F}^{T}\le\vec{s}$, it follows that
$\rdeg_{\vec{t}^{*}}\left[\mathbf{F}^{T},\mathbf{G}^{T}\right]\le\vec{s}$. 
\item The second condition $\sum\vec{t}^{*}\le\xi$ follows from \prettyref{lem:colBasisdegreeBoundByRdegOfRightFactor}.
\item The third condition holds since $\begin{bmatrix}-I~\\
~\mathbf{T}^{T}
\end{bmatrix}$ is a kernel basis with row degrees bounded by $\vec{t}^{*}$.
\end{itemize}
\end{proof}
With a $\left(\left[\mathbf{F}^{T},\mathbf{G}^{T}\right],-\vec{t}^{*}\right)$-kernel
basis $\begin{bmatrix}V\\
\bar{\mathbf{T}}
\end{bmatrix}$ computed, a column basis is then given by $\mathbf{T}~=~\left(\bar{\mathbf{T}}V^{-1}\right)^{T}$.

The complete algorithm for computing a column basis is then given
in Algorithm \ref{alg:colBasis}.

\input{AlgorithmColumnBasis.tex} 
\begin{thm}
\label{thm:columnBasisCost1}A column basis $\mathbf{T}$ of $\mathbf{F}$
can be computed with a cost of $O^{\sim}\left(n^{\omega}s\right)$,
where $s=\xi/n$ is the average column degree of $\mathbf{F}$ as
before. \end{thm}
\begin{proof}
The cost is dominated by the cost of the three kernel basis computations
in the algorithm. The first one is handled by the algorithm from \cite{za2012}
and \prettyref{thm:costGeneral}, while the remaining two are handled
by \prettyref{alg:minimalKernelBasisReverse}, \prettyref{lem:costKernelBasisReverse}
and \prettyref{lem:costOfKernelBasisReversedForLeftFactor}.\end{proof}

