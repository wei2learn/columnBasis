
\section{Preliminaries}

In this paper computational cost is analyzed by bounding the number
of arithmetic operations in the coefficient field $\mathbb{K}$ on
an algebraic random access machine. We assume the cost of multiplying
two polynomial matrices with dimension $n$ and degree $d$ is $O^{\sim}(n^{\omega}d)$
field operations, where the multiplication exponent $\omega$ is assumed
to satisfy $2<\omega\le3$. We refer to the book by \cite{vonzurgathen}
for more details and references about the cost of polynomial multiplication
and matrix multiplication.

In this section we first describe the notations used in this paper,
and then give the basic definitions and properties of {\em shifted
degree}, {\em order basis} and {\em kernel basis} for a matrix
of polynomials. These will be the building blocks used in our algorithm.


\subsection{Notations}

For convenience we adopt the following notations in this paper. 
\begin{description}
\item [{Comparing~Unordered~Lists}] For two lists $\vec{a}\in\mathbb{Z}^{n}$
and $\vec{b}\in\mathbb{Z}^{n}$, let $\bar{a}=\left[\bar{a}_{1},\dots,\bar{a}_{n}\right]\in\mathbb{Z}^{n}$
and $\bar{b}=\left[\bar{b}_{1},\dots,\bar{b}_{n}\right]\in\mathbb{Z}^{n}$
be the lists consists of the entries of $\vec{a}$ and $\vec{b}$
but sorted in increasing order. 
\[
\begin{cases}
\vec{a}\ge\vec{b} & \mbox{if }\bar{a}_{i}\ge\bar{b}_{i}\mbox{ for all }i\in\left[1,\dots n\right]\\
\vec{a}\le\vec{b} & \mbox{if }\bar{a}_{i}\le\bar{b}_{i}\mbox{ for all }i\in\left[1,\dots n\right]\\
\vec{a}>\vec{b} & \mbox{if }\vec{a}\ge\vec{b}\mbox{ and }\bar{a}_{j}>\bar{b}_{j}\mbox{ for at least one }j\in\left[1,\dots n\right]\\
\vec{a}<\vec{b} & \mbox{if }\vec{a}\le\vec{b}\mbox{ and }\bar{a}_{j}<\bar{b}_{j}\mbox{ for at least one }j\in\left[1,\dots n\right].
\end{cases}
\]

\item [{Summation~Notation}] For a list $\vec{a}=\left[a_{1},\dots,a_{n}\right]\in\mathbb{Z}^{n}$,
we write $\sum\vec{a}$ without index to denote the summation of all
entries in $\vec{a}$. 
\item [{}]~
\item [{Uniformly~Shift~a~List}] For a list $\vec{a}=\left[a_{1},\dots,a_{n}\right]\in\mathbb{Z}^{n}$
and $c\in\mathbb{Z}$, we write $\vec{a}+c$ to denote $\vec{a}+\left[c,\dots,c\right]=\left[a_{1}+c,\dots,a_{n}+c\right]$,
and similarly for $-$.
\item [{Compare~a~List~with~a~Integer}] For a list $\vec{a}=\left[a_{1},\dots,a_{n}\right]\in\mathbb{Z}^{n}$
and $c\in\mathbb{Z}$, we write $\vec{a}<c$ to denote $\vec{a}<\left[c,\dots,c\right]$,
and similarly for $>,\le,\ge,=$.
\end{description}

\subsection{Shifted Degrees}

Our methods depend extensively on the concept of {\em shifted}
degrees of polynomial matrices \cite{BLV:1999}. For a column vector
$\mathbf{p}=\left[p_{1},\dots,p_{n}\right]^{T}$ of univariate polynomials
over a field $\mathbb{K}$, its column degree, denoted by $\cdeg\mathbf{p}$,
is the maximum of the degrees of the entries of $\mathbf{p}$, that
is, 
\[
\cdeg\mathbf{p}=\max_{1\le i\le n}\deg p_{i}.
\]
The \emph{shifted column degree} generalizes this standard column
degree by taking the maximum after shifting the degrees by a given
integer vector that is known as a \emph{shift}. More specifically,
the shifted column degree of $\mathbf{p}$ with respect to a shift
$\vec{s}=\left[s_{1},\dots,s_{n}\right]\in\mathbb{Z}^{n}$, or the
\emph{$\vec{s}$-column degree} of $\mathbf{p}$ is 
\[
\cdeg_{\vec{s}}\mathbf{p}=\max_{1\le i\le n}[\deg p_{i}+s_{i}]=\deg(x^{\vec{s}}\cdot\mathbf{p}),
\]
where 
\[
x^{\vec{s}}=\diag\left(x^{s_{1}},x^{s_{2}},\dots,x^{s_{n}}\right)
\]
For a matrix $\mathbf{P}$, we use $\cdeg\mathbf{P}$ and $\cdeg_{\vec{s}}\mathbf{P}$
to denote respectively the list of its column degrees and the list
of its shifted $\vec{s}$-column degrees. When $\vec{s}=\left[0,\dots,0\right]$,
the shifted column degree specializes to the standard column degree.
The shifted row degree of a row vector \textbf{$\mathbf{q}=\left[q_{1},\dots,q_{n}\right]$}
is defined similarly as 
\[
\rdeg_{\vec{s}}\mathbf{q}=\max_{1\le i\le n}[\deg q_{i}+s_{i}]=\deg(\mathbf{q}\cdot x^{\vec{s}}).
\]


The shifted degrees have been used previously in polynomial matrix
computations and to generalize matrix normal forms \cite{BLV:jsc06}.
The shifted column degree is equivalent to the notion of \emph{defect}
commonly used in the literature.

The usefulness of the shifted degrees can be seen from their applications
in polynomial matrix computation problems \cite{ZL2012,za2012}. One
of its uses is illustrated by the following lemma from \cite[Chapter 2]{zhou:phd2012},
which can be viewed as a stronger version of the predictable-degree
property \cite{kailath:1980}. 
\begin{lem}
\label{lem:predictableDegree}Let $\mathbf{A}\in\mathbb{K}\left[x\right]^{m\times n}$
be a $\vec{u}$-column reduced matrix with no zero columns and with
$\cdeg_{\vec{u}}\mathbf{A}=\vec{v}$. Then a matrix $\mathbf{B}\in\mathbb{K}\left[x\right]^{n\times k}$
has $\vec{v}$-column degrees $\cdeg_{\vec{v}}\mathbf{B}=\vec{w}$
if and only if $\cdeg_{\vec{u}}\left(\mathbf{A}\mathbf{B}\right)=\vec{w}$. 
\end{lem}
The following lemma from \cite[Chapter 2]{zhou:phd2012} describes
a relationship between shifted column degrees and shifted row degrees.
\begin{lem}
\label{lem:columnDegreesRowDegreesSymmetry}A matrix $\mathbf{A}\in\mathbb{K}\left[x\right]^{m\times n}$
has $\vec{u}$-column degrees bounded by $\vec{v}$ if and only if
its $-\vec{v}$-row degrees are bounded by $-\vec{u}$. 
\end{lem}
Another essential tool needed in our computation, also based on the
use of the shifted degrees, is the efficient multiplication of matrices
with unbalanced degrees \cite[Theorem 3.7]{za2012}. 
\begin{thm}
\label{thm:multiplyUnbalancedMatrices} Let $\mathbf{A}\in\mathbb{K}\left[x\right]^{m\times n}$
with $m\le n$, $\vec{s}\in\mathbb{Z}^{n}$ a shift with entries bounding
the column degrees of $\mathbf{A}$ and $\xi$, a bound on the sum
of the entries of $\vec{s}$. Let $\mathbf{B}\in\mathbb{K}\left[x\right]^{n\times k}$
with $k\in O\left(m\right)$ and the sum $\theta$ of its $\vec{s}$-column
degrees satisfying $\theta\in O\left(\xi\right)$. Then we can multiply
$\mathbf{A}$ and $\mathbf{B}$ with a cost of $O^{\sim}(n^{2}m^{\omega-2}s)\subset O^{\sim}(n^{\omega}s)$,
where $s=\xi/n$ is the average of the entries of $\vec{s}$. 
\end{thm}

\subsection{Order Basis}

Let $\mathbb{K}$ be a field, $\mathbf{F}\in\mathbb{K}\left[\left[x\right]\right]^{m\times n}$
a matrix of power series and $\vec{\sigma}=\left[\sigma_{1},\dots,\sigma_{m}\right]$
a vector of non-negative integers. 
\begin{defn}
We say a vector of polynomials $\mathbf{p}\in\mathbb{K}\left[x\right]^{n\times1}$
has \emph{order} $\left(\mathbf{F},\vec{\sigma}\right)$ (or \emph{order}
$\vec{\sigma}$ with respect to $\mathbf{F}$) if $\mathbf{F}\cdot\mathbf{p}\equiv\mathbf{0}\mod x^{\vec{\sigma}}$,
that is, 
\[
\mathbf{F}\cdot\mathbf{p}=x^{\vec{\sigma}}\mathbf{r}
\]
for some $\mathbf{r}\in\mathbb{K}\left[\left[x\right]\right]^{m\times1}$.
If $\vec{\sigma}=\left[\sigma,\dots,\sigma\right]$ has entries uniformly
equal to $\sigma$, then we say that $\mathbf{p}$ has order $\left(\mathbf{F},\sigma\right).$
The set of all order $\left(\mathbf{F},\vec{\sigma}\right)$ vectors
is a free $\mathbb{K}\left[x\right]$-module denoted by $\left\langle \left(\mathbf{F},\vec{\sigma}\right)\right\rangle $. 
\end{defn}
An order basis for $\mathbf{F}$ and $\vec{\sigma}$ is simply a basis
for the $\mathbb{K}\left[x\right]$-module $\left\langle \left(\mathbf{F},\vec{\sigma}\right)\right\rangle $.
We again represent order bases using matrices, whose columns are the
basis elements. We only work with those order bases having minimal
or shifted minimal degrees (also referred to as a reduced order basis
in \cite{BL1997}), that is, their column degrees or shifted column
degrees are the smallest possible among all bases of the module. A
discussion on such minimality and its existence can be found in \cite[Chapter 2]{zhou:phd2012}.

An order basis \cite{BeLa94,BL1997} $\mathbf{P}$ of $\mathbf{F}$
with order $\vec{\sigma}$ and shift $\vec{s}$, or simply an $\left(\mathbf{F},\vec{\sigma},\vec{s}\right)$-basis,
is a basis for the module $\left\langle \left(\mathbf{F},\vec{\sigma}\right)\right\rangle $
having minimal $\vec{s}$-column degrees. If $\vec{\sigma}=\left[\sigma,\dots,\sigma\right]$
is uinform then we simply write $\left(\mathbf{F},\sigma,\vec{s}\right)$-basis.
The precise definition of an $\left(\mathbf{F},\vec{\sigma},\vec{s}\right)$-basis
is as follows. 
\begin{defn}
\label{def:orderBasis}A polynomial matrix $\mathbf{P}$ is an order
basis of $\mathbf{F}$ of order $\vec{\sigma}$ and shift $\vec{s}$,
denoted by $\left(\mathbf{F},\vec{\sigma},\vec{s}\right)$-basis,
if the following properties hold: 
\begin{enumerate}
\item $\mathbf{P}$ is a nonsingular matrix of dimension $n$ and is $\vec{s}$-column
reduced. 
\item $\mathbf{P}$ has order $\left(\mathbf{F},\vec{\sigma}\right)$ (or
equivalently, each column of $\mathbf{P}$ is in $\left\langle (\mathbf{F},\vec{\sigma})\right\rangle $). 
\item Any $\mathbf{q}\in\left\langle \left(\mathbf{F},\vec{\sigma}\right)\right\rangle $
can be expressed as a linear combination of the columns of $\mathbf{P}$,
given by $\mathbf{P}^{-1}\mathbf{q}$. 
\end{enumerate}
\end{defn}
%\begin{comment}
%Note that the module $\left\langle \left(\mathbf{F},\vec{\sigma}\right)\right\rangle $
%does not depend on the shift $\vec{s}$. 
%\end{comment}


Note that any pair of $\left(\mathbf{F},\vec{\sigma},\vec{s}\right)$-bases
$\mathbf{P}$ and $\mathbf{Q}$ are column bases of each other and
are unimodularly equivalent.

We will need to compute order bases with unbalanced shifts using Algorithm
2 from \cite{za2009}. This computation can be done efficiently as
given by the following result from \cite{za2009}.
\begin{thm}
\label{thm:unbalancedOrderBasisCost}If the shift $\vec{s}$ satisfies
$\vec{s}\le0$ and $-\sum\vec{s}\le m\sigma$, then a $\left(\mathbf{F},\sigma,\vec{s}\right)$-basis
can be computed with cost of $O^{\sim}(n^{\omega}d)$ field operations,
where $d=m\sigma/n$. 
\end{thm}

\subsection{Kernel Bases}

The kernel of $\mathbf{F}\in\mathbb{K}\left[x\right]^{m\times n}$
is the $\mathbb{F}\left[x\right]$-module 
\[
\left\{ \mathbf{p}\in\mathbb{K}\left[x\right]^{n}~|~\mathbf{F}\mathbf{p}=0\right\} 
\]
with a kernel basis of $\mathbf{F}$ being a basis of this module.
Kernel bases are closely related to order bases, as can be seen from
the following definitions. 
\begin{defn}
\label{def:kernelBasis}Given $\mathbf{F}\in\mathbb{K}\left[x\right]^{m\times n}$,
a polynomial matrix $\mathbf{N}\in\mathbb{K}\left[x\right]^{n\times*}$
is a (right) kernel basis of $\mathbf{F}$ if the following properties
hold: 
\begin{enumerate}
\item $\mathbf{N}$ is full-rank. 
\item $\mathbf{N}$ satisfies $\mathbf{F}\cdot\mathbf{N}=0$. 
\item Any $\mathbf{q}\in\mathbb{K}\left[x\right]^{n}$ satisfying $\mathbf{F}\mathbf{q}=0$
can be expressed as a linear combination of the columns of $\mathbf{N}$,
that is, there exists some polynomial vector $\mathbf{p}$ such that
$\mathbf{q}=\mathbf{N}\mathbf{p}$. 
\end{enumerate}
\end{defn}
Any pair of kernel bases $\mathbf{N}$ and $\mathbf{M}$ of $\mathbf{F}$
are column bases of each other and are unimodularly equivalent.

A $\vec{s}$-minimal kernel basis of $\mathbf{F}$ is just a kernel
basis that is $\vec{s}$-column reduced. 
\begin{defn}
Given $\mathbf{F}\in\mathbb{K}\left[x\right]^{m\times n}$, a polynomial
matrix $\mathbf{N}\in\mathbb{K}\left[x\right]^{n\times*}$ is a $\vec{s}$-minimal
(right) kernel basis of $\mathbf{F}$ if\textbf{ $\mathbf{N}$} is
a kernel basis of $\mathbf{F}$ and $\mathbf{N}$ is $\vec{s}$-column
reduced. We also call a $\vec{s}$-minimal (right) kernel basis of
$\mathbf{F}$ a $\left(\mathbf{F},\vec{s}\right)$-kernel basis. 

We will need to use the following bound on the sizes of kernel bases
from \cite{za2012}.\end{defn}
\begin{thm}
\label{thm:boundOfSumOfShiftedDegreesOfKernelBasis}Suppose $\mathbf{F}\in\mathbb{K}\left[x\right]^{m\times n}$
and $\vec{s}\in\mathbb{Z}_{\ge0}^{n}$ is a shift with entries bounding
the corresponding column degrees of $\mathbf{F}$. Then the sum of
the $\vec{s}$-column degrees of any $\vec{s}$-minimal kernel basis
of $\mathbf{F}$ is bounded by $\sum\vec{s}$.
\end{thm}
We will also need the following result from \cite{za2012} to compute
kernel bases by rows. 
\begin{thm}
\label{thm:continueComputingKernelBasisByRows}Let $\mathbf{G}=\left[\mathbf{G}_{1}^{T},\mathbf{G}_{2}^{T}\right]^{T}\in\mathbb{K}\left[x\right]^{m\times n}$
and $\vec{t}\in\mathbb{Z}^{n}$ a shift vector. If $\mathbf{N}_{1}$
is a $\left(\mathbf{G}_{1},\vec{t}\right)$-kernel basis with $\vec{t}$-column
degrees $\vec{u}$, and $\mathbf{N}_{2}$ is a $\left(\mathbf{G}_{2}\mathbf{N}_{1},\vec{u}\right)$-kernel
basis with $\vec{u}$-column degrees $\vec{v}$, then $\mathbf{N}_{1}\mathbf{N}_{2}$
is a $\left(\mathbf{G},\vec{t}\right)$-kernel basis $\vec{t}$-column
degrees $\vec{v}$. 
\end{thm}
Also recall the cost of kernel basis computation from \cite{za2012}.
\begin{thm}
\label{thm:costGeneral} Given an input matrix $\mathbf{F}\in\mathbb{K}\left[x\right]^{m\times n}$.
Let $\vec{s}=\cdeg\mathbf{F}$ and $s=\sum\vec{s}/n$ be the average
column degree of $\mathbf{F}$. Then a $\left(\mathbf{F},\vec{s}\right)$-kernel
basis can be computed with a cost of $O^{\sim}\left(n^{\omega}s\right)$
field operations.\end{thm}

