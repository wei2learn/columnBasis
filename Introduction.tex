
\section{\label{sec:Matrix-GCD}Introduction}

In this paper, we consider the problem of efficiently computing a
column basis of a polynomial matrix $\mathbf{F}\in\mathbb{K}\left[x\right]^{m\times n}$
with $n\ge m$. A column basis of $\mathbf{F}$ is a basis for the
$\mathbb{K}\left[x\right]$-module 
\[
\left\{ \mathbf{F}\mathbf{p}~|~\mathbf{p}\in\mathbb{K}\left[x\right]^{n}~\right\} ~.
\]
Such a basis can be represented as a full rank matrix $\mathbf{T}\in\mathbb{K}\left[x\right]^{m\times r}$
whose columns are the basis elements and generate the same $\mathbb{K}\left[x\right]$-module
as generated by the columns of $\mathbf{F}$. A column basis is not
unique and indeed any column basis right multiplied by a unimodular
matrix gives another column basis. As a result, a column basis can
have arbitrary high degree. In this paper, the column basis we compute
has column degrees bounded by the largest column degrees of the input
matrix.

Column bases are fundamental operations in polynomial matrix algebra.
As an example, when the row dimension $m=1$, finding a column basis
coincides with finding a greatest common divisor (GCD) of all the
polynomials in the matrix. Similarly, the nonzero columns of column
reduced forms, Popov normal forms, and Hermite normal forms are all
column bases satisfying additional degree constraints. A column reduced
form gives a special column basis whose column degrees are the smallest
possible, while Popov and Hermite forms are special column reduced
or shifted column reduced forms satisfying additional conditions that
make them unique. Efficient column basis computation is thus useful
for fast computation for such core procedures as determining matrix
GCDs \cite{BL2000}, column reduced forms \cite{BVP:1988} and Popov
forms \cite{villard96} of $\mathbf{F}$ with any dimension and rank.
Column basis computation in this paper also provides a deterministic
alternative to randomized lattice compression \cite{li:2006,storjohann-villard:2005}.

While column bases are produced by column reduced form, Popov form,
and Hermite form, and much work has been done on computing these forms
\cite{bcl:2006,beelen:1988,Giorgi2003,GSSV2012,sarkar2011,SS2011},
most of the existing algorithms require square nonsingular input matrices,
which are already column bases to start with. It is however pointed
out in \cite{sarkar2011,SS2011} that randomization can be used to
relax the square nonsingular requirement.

To compute a column basis, we know from \cite{BL1997} that any matrix
polynomial $\mathbf{F}\in\mathbb{K}\left[x\right]^{m\times n}$ can
be unimodularly transformed to a column basis by repeatedly working
with the leading column coefficient matrices. However this method
of computing a column basis can be expensive. Indeed one needs to
work with up to $\sum\vec{s}$ such coefficient matrices, which could
involve up to $\sum\vec{s}$ polynomial matrix multiplications.

In this paper we give a fast, deterministic algorithm for the computation
of a column basis for $\mathbf{F}$ having complexity $O^{\sim}\left(n^{\omega}s\right)$
field operations in $\mathbb{K}$ with $s$ being the average average
column degree of $\mathbf{F}$. Here the soft-$O$ notation is Big-$O$
with log factors removed while $\omega$ is the exponent of matrix
multiplication. Our algorithm works for both rectangular and non-full
column rank matrices with the method depending on kernel computation
of $\mathbf{F}$ along with finding a factorization of the input matrix
polynomial into a column basis and a left kernel of the right kernel.
Finding the right and left kernel then makes use of the fast kernel
and order basis algorithms from \cite{za2012} and \cite{za2009},
respectively.

The remainder of this paper is as follows. Basic definitions and preliminary
results on both kernel and order bases are given in the next section.
Section 3 provides the matrix factorization form of our input polynomial
matrix that forms the core of our procedure, with a column basis being
the left factor, and the right factor is a left kernel basis of a
right kernel basis of the input matrix. Section 4 provides an algorithm
for fast computation of a left kernel making use of order bases computation
with unbalanced shift. The column basis algorithm is given in Section
5 with the following section giving details on how the methods can
be improved when the number of columns is significantly larger than
the number of rows.  The paper ends with a conclusion along with
topics for future research.
