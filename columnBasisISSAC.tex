%% LyX 2.0.4 created this file.  For more info, see http://www.lyx.org/.
%% Do not edit unless you really know what you are doing.
\documentclass[twoside,english]{sig-alternate}
\usepackage[T1]{fontenc}
\usepackage[latin9]{inputenc}
\usepackage{color}
\usepackage{verbatim}
\usepackage{prettyref}
\usepackage{float}
\usepackage{amsmath}
\usepackage{amssymb}
\usepackage[authoryear]{natbib}

\makeatletter

%%%%%%%%%%%%%%%%%%%%%%%%%%%%%% LyX specific LaTeX commands.
\floatstyle{ruled}
\newfloat{algorithm}{tbp}{loa}
\providecommand{\algorithmname}{Algorithm}
\floatname{algorithm}{\protect\algorithmname}

%%%%%%%%%%%%%%%%%%%%%%%%%%%%%% Textclass specific LaTeX commands.
\floatstyle{ruled}
\newfloat{algorithm}{tbp}{loa}
\floatname{algorithm}{Algorithm}
\usepackage{algorithmic}
\newcommand{\forbody}[1]{ #1 \ENDFOR }
\newcommand{\ifbody}[1]{ #1  \ENDIF}
\newcommand{\whilebody}[1]{ #1  \ENDWHILE}
\renewcommand{\algorithmicprint}{\textbf{draw}}
\renewcommand{\algorithmicrequire}{\textbf{Input:}}
\renewcommand{\algorithmicensure}{\textbf{Output:}}

 \newtheorem{thm}{Theorem}[section]
 \newtheorem{lem}[thm]{Lemma}
 \newtheorem{defn}[thm]{Definition}
 \newtheorem{cor}[thm]{Corollary}
 \newtheorem{rem}[thm]{Remark}

%%%%%%%%%%%%%%%%%%%%%%%%%%%%%% User specified LaTeX commands.
%\usepackage{yjsco}\journal{JournalofSymbolicComputation}

\renewcommand{\algorithmicrequire}{\textbf{Input:}}\renewcommand{\algorithmicensure}{\textbf{Output:}}
%\renewcommand{\algorithmicensure}{\textbf{if}}\renewcommand{\algorithmicensure}{\textbf{Uses:}}

\def\diag{\mbox{diag}}\def\cdeg{\qopname\relax n{cdeg}}
\def\MM{\qopname\relax n{MM}}\def\M{\qopname\relax n{M}}\def\ord{\qopname\relax n{ord}}

\def\StorjohannTransform{\qopname\relax n{StorjohannTransform}}\def\TransformUnbalanced{\qopname\relax n{TransformUnbalanced}}\def\rowDimension{\qopname\relax n{rowDimension}}\def\columnDimension{\qopname\relax n{columnDimension}}\DeclareMathOperator{\re}{rem}\DeclareMathOperator{\coeff}{coeff}\DeclareMathOperator{\lcoeff}{lcoeff}\def\mab{\qopname\relax n{orderBasis}}\def\mmab{\qopname\relax n{FastBasis}}\def\umab{\qopname\relax n{UnbalancedFastBasis}}\newcommand{\bb}{\\}
\def\mnb{\qopname\relax n{MinimalNullspaceBasis ~ }}
\DeclareMathOperator{\mnbr}{minimaKernelBasisReversed}
\def\rdeg{\qopname\relax n{rdeg}}
\DeclareMathOperator{\colBasis}{colBasis}


\makeatother
\newcommand{\arne}[1]{{\color{blue}\it {\bf Arne:} #1 }}
\newcommand{\wei}[1]{{\color{red}\it {\bf Wei:} #1}}
\newcommand{\george}[1]{{\color{green}\it {\bf George:} #1}}
\def\newblock{\hskip .11em plus .33em minus .07em}

\makeatother

\usepackage{babel}
\begin{document}
% --- Author Metadata here ---


%\conferenceinfo{ISSAC'09,} {July 28--31, 2009, Seoul, Republic of Korea.} 
%\CopyrightYear{2009}
%\crdata{978-1-60558-609-0/09/07} 

\numberofauthors{1}


\author{\alignauthor Wei Zhou and George Labahn\\
 \affaddr{Cheriton School of Computer Science}\\
 \affaddr{University of Waterloo}, \\
 \affaddr{Waterloo, Ontario, Canada}\\
 \email{\{w2zhou,glabahn\}@uwaterloo.ca} }


\date{{\normalsize \today \quad{}:: \timeofday}}


\title{Computing Column Bases}
\maketitle
\begin{abstract}
Given a matrix of univariate polynomials over a field $\mathbb{K}$,
its columns generates a $\mathbb{K}\left[x\right]$-module consisting
of all $\mathbb{K}\left[x\right]$-linear combination of these columns.
We call any basis of this module a column basis of the given matrix.
In this paper we present a deterministic algorithm for the computation
of a column basis of an $m\times n$ input matrix with $m\le n$.
If $s$ is the average column degree of the input matrix, this algorithm
computes a column basis of with a cost of $O^{\sim}\left(n^{\omega}s\right)$
field operations in $\mathbb{K}$. Here the soft-$O$ notation is
Big-$O$ with log factors removed while $\omega$ is the exponent
of matrix multiplication. Note that the average column degree $s$
is bounded by the commonly used matrix degree that is also the maximum
column degree of the input matrix.
\end{abstract}
% A category with the (minimum) three required fields
%\category{I.1.2}{Symbolic and Algebraic Manipulation}{Algorithms}%[Algebraic algorithms]
%A category including the fourth, optional field follows...
%\category{F.2.2}{Analysis of Algorithms and }{Nonnumerical Algorithms and Problems}%[Computations on discrete structures]

%\terms{Algorithms, Theory}

%\keywords{Order basis, Complexity}

\vspace{1mm} \noindent {\bf Categories and Subject Descriptors:} I.1.2 {[{Symbolic and Algebraic Manipulation}]}: {Algorithms}; F.2.2 {[{Analysis of Algorithms and Problem Complexity}]}: {Nonnumerical Algorithms and Problems}

\vspace{1mm} \noindent {\bf General Terms:} Algorithms, Theory

\vspace{1mm} \noindent {\bf Keywords:} Nullspace basis, Complexity


\section{Introduction}

Let $\mathbf{F}\in\mathbb{K}\left[x\right]^{m\times n}$ be a matrix
of polynomials over a field $\mathbb{K}$. The set 
\[
\left\{ \mathbf{p}\in\mathbb{K}\left[x\right]^{n}~|~\mathbf{F}\mathbf{p}=0\right\} 
\]
forms a $\mathbb{K}[x]$-module. We call any basis of this module
a column basis of $\mathbf{F}$. When the row dimension $m=1$, a
column basis becomes the standard polynomial GCD. We represent column
bases using matrices, with columns being the basis elements.

GCD is one of the most fundamental subject in algebra and computer
algebra and has been extensively studied. As a generalization of polynomial
GCD and matrix GCD, column bases are also fundamental in polynomial
matrix computations. Column basis computation provides a deterministic
alternative to randomized lattice compression \citep{li:2006,storjohann-villard:2005}.
It also helps us to deterministically compute matrix determinants
and Hermite normal forms, and to deterministically compute column
reduced forms and Popov normal forms for matrices of any dimension
and any rank\citep{zhou:phd2012}. 

Note that column bases are not unique. A column basis multiplied by
any unimodular matrix gives another column basis. As a result, a column
basis can have arbitrary high degree. In this paper, we compute a
column basis with column degrees bounded by the largest column degrees
of the input matrix. In fact, the nonzero columns of column reduced
forms, Popov normal forms, and Hermite normal forms are all column
bases satisfying additional degree constraints. A column reduced form
gives a special column basis whose column degrees are the smallest
possible. The Popov form is a special unique column reduced form satisfying
additional conditions. The Hermite normal form provides another special
unique column basis that is triangular and has a different type of
minimal column degrees.


\section{Preliminaries}

The computational cost in this paper is analyzed by bounding the number
of arithmetic operations in the coefficient field $\mathbb{K}$ on
an algebraic random access machine. We assume the cost of multiplying
two polynomial matrices with dimension $n$ and degree $d$ is $O^{\sim}(n^{\omega}d)$
field operations, where the multiplication exponent $\omega$ is assumed
to satisfy $2<\omega\le3$. We refer to the book by \citet{vonzurgathen}
for more details and reference about the cost of polynomial multiplication
and matrix multiplication.

In this paper, we consider the problem of computing a column basis
of an input matrix $\mathbf{F}\in\mathbb{K}\left[x\right]^{m\times n}$,
assuming its row dimension is no more than its column dimension and
its column degrees are bounded by the entries of a list $\vec{s}\in\mathbb{Z}^{n}$.
A column basis $\mathbf{T}\in\mathbb{K}\left[x\right]^{m\times r}$
of $\mathbf{F}$ has the same rank as $\mathbf{F}$ and has full column
rank, since the basis elements are linearly independent.

Any matrix $\mathbf{F}$ in $\mathbb{K}\left[x\right]^{m\times n}$
can be unimodularly transformed to $\left[\mathbf{T},0\right]$ with
a full column rank $\mathbf{T}$ that is also a column basis of $\mathbf{F}$,
as we know $\mathbf{F}$ can be unimodularly transformed to a column
reduced form. One way to do so is to compute a column reduced form
by repeatedly working with the leading column coefficient matrices.
More specifically, if the nonzero columns of the leading column coefficient
matrix $F$ of $\mathbf{F}$ is not full rank, then let $\mathbf{f}$
be the highest degree column from $\mathbf{F}$ such that its leading
column coefficient vector $f$ is not linearly independent from the
remaining columns. Therefore, we can eliminate $f$ against other
columns by a column operation, or equivalently, a unimodular transformation.
It follows that the same column operation on $\mathbf{F}$ lowers
the degree of $\mathbf{f}$.

But computing a column basis this way can be expensive, as we need
to work with up to $\xi=\sum\vec{s}$ such coefficient matrices, which
would involve up to $\xi$ polynomial matrix multiplications.  To
efficiently compute a column basis, we make use of order basis computations
and kernel basis computations. Efficient algorithms for these have
been developed in \citep{ZL2012,za2012,zhou:phd2012}. As in the computation
of order bases and kernel bases, we also make extensive use of shifted
degrees. 


\subsection{Shifted Degrees}

For a column vector $\mathbf{p}=\left[p_{1},\dots,p_{n}\right]^{T}$
of univariate polynomials over a field $\mathbb{K}$, its column degree,
denoted by $\cdeg\mathbf{p}$, is just the maximum of the degrees
of the entries of $\mathbf{p}$, that is, 
\[
\cdeg\mathbf{p}=\max_{1\le i\le n}\deg p_{i}.
\]
The \emph{shifted column degree} generalizes this standard column
degree by taking the maximum after shifting the degrees by a given
integer vector that is known as a \emph{shift}. More specifically,
the shifted column degree of $\mathbf{p}$ with respect to a shift
$\vec{s}=\left[s_{1},\dots,s_{n}\right]\in\mathbb{Z}^{n}$, or the
\emph{$\vec{s}$-column degree} of $\mathbf{p}$ is

\[
\cdeg_{\vec{s}}\mathbf{p}=\max_{1\le i\le n}[\deg p_{i}+s_{i}]=\deg(x^{\vec{s}}\cdot\mathbf{p}),
\]
 where 
\[
x^{\vec{s}}=\diag\left(\left[x^{s_{1}},x^{s_{2}},\dots,x^{s_{n}}\right]\right)=\begin{bmatrix}x^{s_{1}}\\
 & x^{s_{2}}\\
 &  & \ddots\\
 &  &  & x^{s_{1}}
\end{bmatrix}.
\]
For a matrix $\mathbf{P}$, we use $\cdeg\mathbf{P}$ and $\cdeg_{\vec{s}}\mathbf{P}$
to denote respectively the list of its column degrees and the list
of its shifted $\vec{s}$-column degrees. When $\vec{s}=\left[0,\dots,0\right]$,
the shifted column degree specializes to the standard column degree.
The shifted row degree of a row vector \textbf{$\mathbf{q}$ }is defined
in the same way.
\[
\rdeg_{\vec{s}}\mathbf{q}=\max_{1\le i\le n}[\deg q_{i}+s_{i}]=\deg(\mathbf{q}\cdot x^{\vec{s}}).
\]


The shifted degrees have been used previously in polynomial matrix
computations and to generalize matrix normal forms \citep{BLV:jsc06}.
The shifted column degree is equivalent to the notion of \emph{defect}
commonly used in the literature.

The usefulness of the shifted degrees can be seen from their applications
in polynomial matrix computation problems \citep{za2012,zhou:phd2012,ZL2012}.
One of its uses is illustrated by the following lemma from Chapter
2 in \citep{zhou:phd2012}, which can be viewed as a stronger version
of the predictable-degree property \citep{kailath:1980}.
\begin{lem}
\label{lem:predictableDegree}If a matrix $\mathbf{A}\in\mathbb{K}\left[x\right]^{m\times n}$
is a $\vec{u}$-column reduced matrix with no zero columns and with
$\cdeg_{\vec{u}}\mathbf{A}=\vec{v}$, then a matrix $\mathbf{B}\in\mathbb{K}\left[x\right]^{n\times k}$
has $\vec{v}$-column degrees $\cdeg_{\vec{v}}\mathbf{B}=\vec{w}$
if and only if $\cdeg_{\vec{u}}\left(\mathbf{A}\mathbf{B}\right)=\vec{w}$.
\end{lem}
Another essential tool needed in our computation is the efficient
multiplication of matrices with unbalanced degrees \citep{za2012,zhou:phd2012}.
\begin{thm}
\label{thm:multiplyUnbalancedMatrices} Let $\mathbf{A}\in\mathbb{K}\left[x\right]^{m\times n}$,
$\vec{s}$ a shift with entries bounding the column degrees of $\mathbf{A}$
and $\xi$, a bound on the sum of the entries of $\vec{s}$. Let $\mathbf{B}\in\mathbb{K}\left[x\right]^{n\times k}$
with $k\in O\left(m\right)$ and the sum $\theta$ of its $\vec{s}$-column
degrees satisfying $\theta\in O\left(\xi\right)$. Then we can multiply
$\mathbf{A}$ and $\mathbf{B}$ with a cost of $O^{\sim}(nm^{\omega-2}\xi)$.
\end{thm}

\subsection{Order Basis}

Let $\mathbb{K}$ be a field, $\mathbf{F}\in\mathbb{K}\left[\left[x\right]\right]^{m\times n}$
a matrix of power series and $\vec{\sigma}=\left[\sigma_{1},\dots,\sigma_{m}\right]$
a vector of non-negative integers. 
\begin{defn}
We say a vector of polynomials $\mathbf{p}\in\mathbb{K}\left[x\right]^{n\times1}$
has \emph{order} $\left(\mathbf{F},\vec{\sigma}\right)$ (or \emph{order}
$\vec{\sigma}$ with respect to $\mathbf{F}$) if $\mathbf{F}\cdot\mathbf{p}\equiv\mathbf{0}\mod x^{\vec{\sigma}}$,
that is, 
\[
\mathbf{F}\cdot\mathbf{p}=x^{\vec{\sigma}}\mathbf{r}=\begin{bmatrix}x^{\sigma_{1}}\\
 & \ddots\\
 &  & x^{\sigma_{m}}
\end{bmatrix}\mathbf{r}
\]
 for some $\mathbf{r}\in\mathbb{K}\left[\left[x\right]\right]^{m\times1}$.
If $\vec{\sigma}=\left[\sigma,\dots,\sigma\right]$ is uniform, then
we say that $\mathbf{p}$ has order $\left(\mathbf{F},\sigma\right).$
\begin{comment}
The vector of power series $\mathbf{r}$ is called the order $\left(\mathbf{F},\sigma\right)$-residual
of \textbf{$\mathbf{p}$}. 
\end{comment}
{} The set of all order $\left(\mathbf{F},\vec{\sigma}\right)$ vectors
is a free $\mathbb{K}\left[x\right]$-module denoted by $\left\langle \left(\mathbf{F},\vec{\sigma}\right)\right\rangle $. 
\end{defn}
An order basis for $\mathbf{F}$ and $\vec{\sigma}$ is simply a basis
for the $\mathbb{K}\left[x\right]$-module $\left\langle \left(\mathbf{F},\vec{\sigma}\right)\right\rangle $.
We only work with those order bases having minimal or shifted minimal
degrees (also referred to as a reduced order basis in \citep{BL1997}).



An \emph{order basis} \citep{BeLa94,BL1997} $\mathbf{P}$ of $\mathbf{F}$
with order $\vec{\sigma}$ and shift $\vec{s}$, or simply an $\left(\mathbf{F},\vec{\sigma},\vec{s}\right)$-basis,
is a basis for the module $\left\langle \left(\mathbf{F},\vec{\sigma}\right)\right\rangle $
\begin{comment}
\[
\left\langle \left(\mathbf{F},\vec{\sigma}\right)\right\rangle =\{\mathbf{p}\in\mathbb{K}\left[x\right]^{n\times1}\|\mathbf{F}\cdot\mathbf{p}=x^{\vec{\sigma}}\mathbf{r},\mathbf{r}\in\mathbb{K}[[x]]^{m\times1}\}
\]
\end{comment}
{} having minimal $\vec{s}$-column degrees. If $\vec{\sigma}=\left[\sigma,\dots,\sigma\right]$
are constant vectors then we simply write $\left(\mathbf{F},\sigma,\vec{s}\right)$-basis.
The precise definition of an $\left(\mathbf{F},\vec{\sigma},\vec{s}\right)$-basis
is as follows.
\begin{defn}
\label{def:orderBasis}A polynomial matrix $\mathbf{P}$ is an order
basis of $\mathbf{F}$ of order $\vec{\sigma}$ and shift $\vec{s}$,
denoted by $\left(\mathbf{F},\vec{\sigma},\vec{s}\right)$-basis,
if the following properties hold: 
\begin{enumerate}
\item $\mathbf{P}$ is a nonsingular matrix of dimension $n$ and is $\vec{s}$-column
reduced. 
\item $\mathbf{P}$ has order $\left(\mathbf{F},\vec{\sigma}\right)$ (or
equivalently, each column of $\mathbf{P}$ is in $\left\langle (\mathbf{F},\vec{\sigma})\right\rangle $). 
\item Any $\mathbf{q}\in\left\langle \left(\mathbf{F},\vec{\sigma}\right)\right\rangle $
can be expressed as a linear combination of the columns of $\mathbf{P}$,
given by $\mathbf{P}^{-1}\mathbf{q}$. 
\end{enumerate}
\end{defn}
\begin{comment}
Note that the module $\left\langle \left(\mathbf{F},\vec{\sigma}\right)\right\rangle $
does not depend on the shift $\vec{s}$. 
\end{comment}




Note that any pair of $\left(\mathbf{F},\vec{\sigma},\vec{s}\right)$-bases
$\mathbf{P}$ and $\mathbf{Q}$ are column bases of each other and
are unimodularly equivalent.


\subsection{Kernel Bases}

Recall that the kernel of $\mathbf{F}\in\mathbb{K}\left[x\right]^{m\times n}$
is the $\mathbb{F}\left[x\right]$-module 
\[
\left\{ \mathbf{p}\in\mathbb{K}\left[x\right]^{n}~|~\mathbf{F}\mathbf{p}=0\right\} .
\]
 A kernel basis of $\mathbf{F}$ is just a basis of this module. Kernel
bases are closely related to order bases, as can be seen from the
following definitions. 
\begin{defn}
\label{def:kernelBasis}Given $\mathbf{F}\in\mathbb{K}\left[x\right]^{m\times n}$,
a polynomial matrix $\mathbf{N}\in\mathbb{K}\left[x\right]^{n\times*}$
is a (right) kernel basis of $\mathbf{F}$ if the following properties
hold: 
\begin{enumerate}
\item $\mathbf{N}$ is full-rank. 
\item $\mathbf{N}$ satisfies $\mathbf{F}\cdot\mathbf{N}=0$. 
\item Any $\mathbf{q}\in\mathbb{K}\left[x\right]^{n}$ satisfying $\mathbf{F}\mathbf{q}=0$
can be expressed as a linear combination of the columns of $\mathbf{N}$,
that is, there exists some polynomial vector $\mathbf{p}$ such that
$\mathbf{q}=\mathbf{N}\mathbf{p}$. 
\end{enumerate}
\end{defn}
Again, note that any pair of kernel bases $\mathbf{N}$ and $\mathbf{M}$
of $\mathbf{F}$ are column bases of each other and are unimodularly
equivalent.

An $\vec{s}$-minimal kernel basis of $\mathbf{F}$ is just a kernel
basis that is $\vec{s}$-column reduced.
\begin{defn}
Given $\mathbf{F}\in\mathbb{K}\left[x\right]^{m\times n}$, a polynomial
matrix $\mathbf{N}\in\mathbb{K}\left[x\right]^{n\times*}$ is a $\vec{s}$-minimal
(right) kernel basis of $\mathbf{F}$ if\textbf{ $\mathbf{N}$} is
a kernel basis of $\mathbf{F}$ and $\mathbf{N}$ is $\vec{s}$-column
reduced. We also call a $\vec{s}$-minimal (right) kernel basis of
$\mathbf{F}$ a $\left(\mathbf{F},\vec{s}\right)$-kernel basis.
\end{defn}
We will need the following result from \citep{za2012,zhou:phd2012}
to compute kernel bases by rows.
\begin{thm}
\label{thm:continueComputingNullspaceBasisByRows}Let $\mathbf{G}=\left[\mathbf{G}_{1}^{T},\mathbf{G}_{2}^{T}\right]^{T}\in\mathbb{K}\left[x\right]^{m\times n}$
and $\vec{t}$ a shift vector. If $\mathbf{N}_{1}$ is a $\left(\mathbf{G}_{1},\vec{t}\right)$-kernel
basis with $\vec{t}$-column degrees $\vec{u}$, and $\mathbf{N}_{2}$
is a $\left(\mathbf{G}_{2}\mathbf{N}_{1},\vec{u}\right)$-kernel basis
with $\vec{u}$-column degrees $\vec{v}$, then $\mathbf{N}_{1}\mathbf{N}_{2}$
is a $\left(\mathbf{G},\vec{t}\right)$-kernel basis $\vec{t}$-column
degrees $\vec{v}$.
\end{thm}

\section{Computing Column Bases}

Before discussing the efficient computation of column basis, it is
useful to look at following relationship between column basis, kernel
basis, and unimodular matrix.
\begin{lem}
\label{lem:unimodular_kernel_columnBasis}Given $\mathbf{F}\in\mathbb{K}\left[x\right]^{m\times n}$.
If $\mathbf{U}$ is a unimodular matrix such that $\mathbf{F}\mathbf{U}=\left[0,\mathbf{T}\right]$
gives a full column rank $\mathbf{T}$, then $\mathbf{U}$ can be
separated into two submatrices $\mathbf{U}=\left[\mathbf{U}_{L},\mathbf{U}_{R}\right]$,
where $\mathbf{U}_{L}$ is a kernel basis of $\mathbf{F}$ and $\mathbf{F}\mathbf{U}_{R}=\mathbf{T}$
is a column basis of $\mathbf{F}$. In addition, the kernel basis
$\mathbf{U}_{L}$ can be replaced by any other kernel basis $\mathbf{N}$
of $\mathbf{F}$ and still gives a unimodular matrix $\left[\mathbf{N},\mathbf{U}_{R}\right]$,
which can also be used to unimodularly transform $\mathbf{F}$ to
$\left[0,\mathbf{T}\right]$. \end{lem}
\begin{proof}
The fact that $\mathbf{T}$ is a column basis of $\mathbf{F}$ follows
directly from the unimodular equivalence between $\mathbf{F}$ and
$\left[0,\mathbf{T}\right]$ and the fact that $\mathbf{T}$ is full
rank. It remains to show that $\mathbf{U}_{L}$ is a kernel basis
of $\mathbf{F}$. Since $\mathbf{F}\mathbf{U}_{L}=0$, $\mathbf{U}_{L}$
is generated by any kernel basis $\mathbf{N}$, that is, $\mathbf{U}_{L}=\mathbf{N}\mathbf{C}$
for some polynomial matrix $\mathbf{C}$. Let $r$ be the rank of
$\mathbf{F}$, which is also the column dimension of $\mathbf{T}$
and $\mathbf{U}_{R}$. Then both $\mathbf{N}$ and $\mathbf{U}_{L}$
have column dimension $n-r$. Hence $\mathbf{C}$ is a square $(n-r)\times(n-r)$
matrix. Now the unimodular matrix $\mathbf{U}$ can be factored as
\[
\mathbf{U}=\left[\mathbf{N}\mathbf{C},\mathbf{U}_{R}\right]=\left[\mathbf{N},\mathbf{U}_{R}\right]\begin{bmatrix}\mathbf{C}\\
 & I
\end{bmatrix},
\]
 implying that both factors $\left[\mathbf{N},\mathbf{U}_{R}\right]$
and $\begin{bmatrix}\mathbf{C}\\
 & I
\end{bmatrix}$ are unimodular. Therefore, $\mathbf{C}$ is unimodular and $\mathbf{U}_{L}=\mathbf{N}\mathbf{C}$
is also a kernel basis. Notice that the unimodular matrix $\left[\mathbf{N},\mathbf{U}_{R}\right]$
also transforms $\mathbf{F}$ to $\left[0,\mathbf{T}\right]$.
\end{proof}
\prettyref{lem:unimodular_kernel_columnBasis} gives the following
result for a unimodular matrix and its inverse.
\begin{cor}
\label{cor:unimodular_kernel_columnBasis2}Let $\mathbf{U}=\left[\mathbf{U}_{L},\mathbf{U}_{R}\right]$
be any unimodular matrix with columns separated arbitrarily to $\mathbf{U}_{L}$
and $\mathbf{U}_{R}$. Let its inverse $\mathbf{V}=\begin{bmatrix}\mathbf{V}_{U}\\
\mathbf{V}_{D}
\end{bmatrix}$, where the row dimension of $\mathbf{V}_{U}$ matches the column
dimension of $\mathbf{U}_{L}$. So we have 
\[
\mathbf{V}\mathbf{U}=\begin{bmatrix}\mathbf{V}_{U}\\
\mathbf{V}_{D}
\end{bmatrix}\begin{bmatrix}\mathbf{U}_{L},\mathbf{U}_{R}\end{bmatrix}=\begin{bmatrix}\mathbf{V}_{U}\mathbf{U}_{L} & \mathbf{V}_{U}\mathbf{U}_{R}\\
\mathbf{V}_{D}\mathbf{U}_{L} & \mathbf{V}_{D}\mathbf{U}_{R}
\end{bmatrix}=\begin{bmatrix}I & 0\\
0 & I
\end{bmatrix}.
\]
Then $\mathbf{V}_{U}\mathbf{U}_{L}=I$ is a column basis of $\mathbf{V}_{U}$
and a row basis of $\mathbf{U}_{L}$, while $\mathbf{V}_{D}\mathbf{U}_{R}=I$
is a column basis of $\mathbf{V}_{D}$ and a row basis of of $\mathbf{U}_{R}$.
In addition, $\mathbf{V}_{D}$ and $\mathbf{U}_{L}$ are kernel bases
of each other, while $\mathbf{V}_{U}$ and $\mathbf{U}_{R}$ are kernel
bases of each other.\end{cor}
\begin{proof}
This follows directly from \prettyref{lem:unimodular_kernel_columnBasis},
by taking $\mathbf{F}$ from \prettyref{lem:unimodular_kernel_columnBasis}
to be $\mathbf{V}_{U}$, $\mathbf{V}_{D}$, $\mathbf{U}_{L}^{T}$,
and $\mathbf{U}_{R}^{T}$ here.
\end{proof}
To compute a column basis of $\mathbf{F}$, we use the following procedure.
We first compute a right  kernel basis $\mathbf{N}$ of $\mathbf{F}$.
Then we compute a left  kernel basis $\mathbf{G}$ of $\mathbf{N}$.
This matrix $\mathbf{G}$ is a right factor of $\mathbf{F}$, that
is, $\mathbf{F}=\mathbf{T}\mathbf{G}$ for some $\mathbf{T}\in\mathbb{K}\left[x\right]^{m\times r}$.
Then we can compute the left factor $\mathbf{T}$, which is in fact
a column basis of $\mathbf{F}$.
\begin{lem}
\label{lem:matrixGCD}Given $\mathbf{F}\in\mathbb{K}\left[x\right]^{m\times n}$%
\begin{comment}
 and has full row rank
\end{comment}
. Let $\mathbf{N}\in\mathbb{K}\left[x\right]^{n\times(n-r)}$ be any
right kernel basis of $\mathbf{F}$, and $\mathbf{G}\in\mathbb{K}\left[x\right]^{r\times n}$
be any left kernel basis of $\mathbf{N}$, where $r$ is the rank
of $\mathbf{F}$. Then $\mathbf{F}=\mathbf{T}\mathbf{G}$ for some
$\mathbf{T}\in\mathbb{K}\left[x\right]^{m\times r}$ and $\mathbf{T}$
is a column basis of $\mathbf{F}$.\end{lem}
\begin{proof}
Let the matrix $\mathbf{U}=\begin{bmatrix}\mathbf{U}_{L},\mathbf{U}_{R}\end{bmatrix}$
from \prettyref{cor:unimodular_kernel_columnBasis2} be a unimodular
matrix that transforms $\mathbf{F}$ to a column basis $\mathbf{B}\in\mathbb{K}\left[x\right]^{m\times r}$
of $\mathbf{F}$, where $\mathbf{U}_{L}$ is any right kernel basis
$\mathbf{F}$. From $\mathbf{F}\mathbf{U}=\left[0,\mathbf{B}\right]$,
we get\textbf{ $\mathbf{F}=\left[0,\mathbf{B}\right]\mathbf{U}^{-1}=\mathbf{B}\left[0,I\right]\mathbf{V}=\mathbf{B}\mathbf{V}_{D}$}.
Since $\mathbf{V}_{D}$ is a left kernel basis of\textbf{ $\mathbf{U}_{L}$},
any other left kernel basis $\mathbf{G}$ of $\mathbf{U}_{L}$ is
unimodularly equivalent to $\mathbf{V}_{D}$, that is, $\mathbf{V}_{D}=\mathbf{W}\mathbf{G}$
for some unimodular matrix $\mathbf{W}$. Now $\mathbf{F}=\mathbf{B}\mathbf{W}\mathbf{G}$,
where $\mathbf{BW}=\mathbf{T}$ a column basis of $\mathbf{F}$ since
it is unimodularly equivalent to the column basis $\mathbf{B}$.
\end{proof}


\prettyref{lem:matrixGCD} outlines a procedure for computing a column
basis of $\mathbf{F}$ with three main steps. The first step is to
compute a $\left(\mathbf{F},\vec{s}\right)$-kernel basis $\mathbf{N}$,
which can be efficiently done using the kernel basis computation algorithm
from \citep{za2012,zhou:phd2012}. However, we still need to work
on the second step of computing a $\left(\mathbf{N}^{T},-\vec{s}\right)$-kernel
basis $\mathbf{G}^{T}$ and the third step of computing the column
basis $\mathbf{T}$ from $\mathbf{F}$ and $\mathbf{G}$. Note that
while \prettyref{lem:matrixGCD} does not require the bases computed
to be minimal, working with minimal bases keeps the degrees well-managed
and helps to make the computation efficient.


\section{\label{sec:computeRightFactor}Computing a Right Factor}

Let us now look at the computation of a $\left(\mathbf{N}^{T},-\vec{s}\right)$-kernel
basis $\mathbf{G}^{T}$. For this problem, the kernel basis computation
algorithm from \citep{za2012} does not work well directly, since
the input matrix $\mathbf{N}^{T}$ has nonuniform row degrees and
negative shift. Comparing to the earlier problem of computing a $\left(\mathbf{F},\vec{s}\right)$-kernel
basis $\mathbf{N}$, it is interesting to note that the old output
$\mathbf{N}$ now becomes the new input matrix $\mathbf{N}^{T}$,
while the new output matrix $\mathbf{G}$ has size bounded by $\mathbf{F}$.
In other words, the new input has degrees that matches the old output,
while the new output has degrees bounded by the old input. It is
therefore reasonable to expect that the new problem can be computed
efficiently. However, we need to find some way to work with the more
complicated input degree structure. On the other hand, the simpler
output degree structure makes it easier to apply order basis computation
to compute a $\left(\mathbf{N}^{T},-\vec{s}\right)$-kernel basis. 

To see how order basis computations can be applied here, let us first
extend Lemma 3.3 from \citep{za2012}, which provides a relationship
between order bases and kernel bases, to accommodate our situation
here.
\begin{lem}
\label{lem:orderbasisContainsNullspacebasisGeneralized}Given a matrix
$\mathbf{A}\in\mathbb{K}\left[x\right]^{m\times n}$ and a degree
shift $\vec{u}$ with $\rdeg_{\vec{u}}\mathbf{A}\le\vec{v}$, or equivalently,
$\cdeg_{-\vec{v}}\mathbf{A}\le-\vec{u}$. Let $\mathbf{P}=\left[\mathbf{P}_{1},\mathbf{P}_{2}\right]$
be any $\left(\mathbf{A},\vec{v}+1,-\vec{u}\right)$-basis and $\mathbf{Q}=\left[\mathbf{Q}_{1},\mathbf{Q}_{2}\right]$
be any $(\mathbf{A},-\vec{u})$-kernel basis, where $\mathbf{P}_{1}$
and $\mathbf{Q}_{1}$ contain all columns from $\mathbf{P}$ and $\mathbf{Q}$,
respectively, whose $-\vec{u}$-column degrees are no more than $0$.
Then $\left[\mathbf{P}_{1},\mathbf{Q}_{2}\right]$ is an $(\mathbf{A},-\vec{u})$-kernel
basis, and $\left[\mathbf{Q}_{1},\mathbf{P}_{2}\right]$ is an $\left(\mathbf{A},\vec{v}+\left[1,\dots,1\right],-\vec{u}\right)$-basis.\end{lem}
\begin{proof}
We know $\cdeg_{-\vec{v}}\mathbf{A}\mathbf{P}_{1}\le\cdeg_{-\vec{u}}\mathbf{P}_{1}\le0$,
or equivalently, $\rdeg\mathbf{A}\mathbf{P}_{1}\le\vec{v}$, but it
has order greater than $\vec{v}$, hence $\mathbf{A}\mathbf{P}_{1}=0$.
The result then follows the same reasoning as in the proof of Lemma
3.3 from \citep{za2012}.%
\begin{comment}
We know $\cdeg\mathbf{P}_{1}^{T}\le\vec{u}$ from $\cdeg_{-\vec{u}}\mathbf{P}_{1}\le0$,
hence $\cdeg\mathbf{P}_{1}^{T}\mathbf{A}^{T}\le\cdeg_{\vec{u}}\mathbf{A}^{T}$
by \prettyref{lem:boundOnDegreesOfFA}. Now for each row $\mathbf{a}_{i}$
in $\mathbf{A}$ and its $\vec{u}$-column degree $v_{i}$, we have
\[
\rdeg\mathbf{a}_{i}\mathbf{P}_{1}=\cdeg\mathbf{P}_{1}^{T}\mathbf{a}_{i}^{T}\le\cdeg_{\vec{u}}\mathbf{a}_{i}^{T}=v_{i},
\]
 and $\mathbf{a}_{i}\mathbf{P}_{1}\equiv0\mod x^{v_{i}+1}$, hence
$\mathbf{A}\mathbf{P}_{1}=0$. The result then follows the same reasoning
as in the proof of \prettyref{lem:orderBasisContainsNullspaceBasis}.
\end{comment}

\end{proof}
Now with the help of \prettyref{lem:orderbasisContainsNullspacebasisGeneralized},
let us get back to our problem of computing a $(\mathbf{F},\vec{s})$-kernel
basis.  In fact, we just need to use a special case of \prettyref{lem:orderbasisContainsNullspacebasisGeneralized},
where all the elements of the kernel basis have shifted degrees bounded
by $0$, making the partial kernel basis a complete kernel basis%
\begin{comment}
, which follows from our requirement of using a shift $\vec{s}\ge\cdeg\mathbf{F}$
\end{comment}
.
\begin{lem}
\label{lem:nullspaceBasisInOrderBasis}Let $\mathbf{N}$ be a $(\mathbf{F},\vec{s})$-kernel
basis with $\cdeg_{\vec{s}}\mathbf{N}=\vec{b}$. Let $\mathbf{P}=\left[\mathbf{P}_{1},\mathbf{P}_{2}\right]$
be a $\left(\mathbf{N}^{T},\vec{b}+1,-\vec{s}\right)$-basis, where
$\mathbf{P}_{1}$ consists of all columns $\mathbf{p}$ with $\cdeg_{-\vec{s}}\mathbf{p}\le0$.
\begin{comment}
of $\mathbf{P}$ satisfying $\mathbf{N}^{T}\mathbf{p}=0$. 
\end{comment}
Then $\mathbf{P}_{1}$ is a $(\mathbf{N}^{T},-\vec{s})$-kernel basis. \end{lem}
\begin{proof}
Let the rank of $\mathbf{F}$ be $r$, which is also the column dimension
of any $(\mathbf{N}^{T},-\vec{s})$-kernel basis $\mathbf{G}^{T}$.
Since both $\mathbf{F}$ and $\mathbf{G}$ are in the left kernel
of $\mathbf{N}$, we know $\mathbf{F}$ is generated by $\mathbf{G}$,
and the $-\vec{s}$-row degrees of $\mathbf{G}$ are bounded by the
corresponding $r$ largest $-\vec{s}$-row degrees of $\mathbf{F}$,
which are in turn bounded by $0$ since $\cdeg\mathbf{F}\le\vec{s}$.
Therefore, any $(\mathbf{N}^{T},-\vec{s})$-kernel basis $\mathbf{G}^{T}$
satisfies $\cdeg_{-\vec{s}}\mathbf{G}^{T}\le0$. The result now follows
from \prettyref{lem:orderbasisContainsNullspacebasisGeneralized}.
\end{proof}
We can use \prettyref{thm:continueComputingNullspaceBasisByRows}
to compute a $\left(\mathbf{N}^{T},-\vec{s}\right)$-kernel basis
by rows. If we separate $\mathbf{N}$ to $\mathbf{N}=\left[\mathbf{N}_{1},\mathbf{N}_{2}\right]$
with $\vec{s}$-column degrees $\vec{b}_{1}$, $\vec{b}_{2}$ respectively,
and first compute a $\left(\mathbf{N}_{1}^{T},-\vec{s}\right)$-kernel
basis $\mathbf{Q}_{1}$ with $-\vec{s}$-column degrees $-\vec{s}_{2}$,
and then compute a $\left(\mathbf{N}_{2}^{T}\mathbf{Q}_{1},-\vec{s}_{2}\right)$-kernel
basis $\mathbf{Q}_{2}$, then $\mathbf{Q}_{1}\mathbf{Q}_{2}$ is a
$\left(\mathbf{N}^{T},-\vec{s}\right)$-kernel basis. To compute kernel
bases $\mathbf{Q}_{1}$ and $\mathbf{Q}_{2}$, we can use order basis
computation. However, we need to make sure that the order bases we
compute do contain these kernel bases.
\begin{lem}
\label{lem:nullspaceBasisOfSubsetOfRowsContainedInOrderBasis}Let
$\mathbf{N}=\left[\mathbf{N}_{1},\mathbf{N}_{2}\right]$, with $\vec{s}$-column
degrees $\vec{b}_{1}$, $\vec{b}_{2}$ respectively. Then a $\left(\mathbf{N}_{1}^{T},\vec{b}_{1}+1,-\vec{s}\right)$-basis
contains a $\left(\mathbf{N}_{1}^{T},-\vec{s}\right)$-kernel basis
whose $-\vec{s}$-row degrees are bounded by 0. Let $\mathbf{Q}_{1}$
be this kernel basis, and $-\vec{s}_{2}=\cdeg_{-\vec{s}}\mathbf{Q}_{1}$.
Then a $\left(\mathbf{N}_{2}^{T}\mathbf{Q}_{1},\vec{b}_{2}+1,-\vec{s}_{2}\right)$-basis
contains a $\left(\mathbf{N}_{2}^{T}\mathbf{Q}_{1},-\vec{s}_{2}\right)$-kernel
basis $\mathbf{Q}_{2}$ whose $-\vec{s}$-row degrees are bounded
by 0. The product $\mathbf{Q}_{1}\mathbf{Q}_{2}$ is then a $\left(\mathbf{N}^{T},-\vec{s}\right)$-kernel
basis.\end{lem}
\begin{proof}
To see that a $\left(\mathbf{N}_{1}^{T},\vec{b}_{1}+1,-\vec{s}\right)$-basis
contains a $\left(\mathbf{N}_{1}^{T},-\vec{s}\right)$-kernel basis
whose $-\vec{s}$-row degrees are bounded by 0, we just need to show
that $\rdeg_{-\vec{s}}\mathbf{\bar{Q}}_{1}\le0$ for any $\left(\mathbf{N}_{1}^{T},-\vec{s}\right)$-kernel
basis $\mathbf{\bar{Q}}_{1}$ and then apply \prettyref{lem:orderbasisContainsNullspacebasisGeneralized}.
Note that there exists a polynomial matrix $\bar{\mathbf{Q}}_{2}$
such that $\mathbf{\bar{Q}}_{1}\mathbf{\bar{Q}}_{2}=\bar{\mathbf{G}}$
for any $\left(\mathbf{N}^{T},-\vec{s}\right)$-kernel basis $\bar{\mathbf{G}}$,
as $\bar{\mathbf{G}}$ satisfies $\mathbf{N}_{1}^{T}\bar{\mathbf{G}}=0$
and is therefore generated by the $\left(\mathbf{N}_{1}^{T},-\vec{s}\right)$-kernel
basis $\bar{\mathbf{Q}}_{1}$. If $\rdeg_{-\vec{s}}\mathbf{\bar{Q}}_{1}\nleq0$,
then \prettyref{lem:predictableDegree} forces $\rdeg_{-\vec{s}}\left(\bar{\mathbf{Q}}_{1}\bar{\mathbf{Q}}_{2}\right)=\rdeg_{-\vec{s}}\bar{\mathbf{G}}\nleq0$,
a contradiction. 

As before, to see that a $\left(\mathbf{N}_{2}^{T}\mathbf{Q}_{1},\vec{b}_{2}+1,-\vec{s}_{2}\right)$-basis
contains a $\left(\mathbf{N}_{2}^{T}\mathbf{Q}_{1},-\vec{s}_{2}\right)$-kernel
basis whose $-\vec{s}$-row degrees are no more than 0, we can just
show $\rdeg_{-\vec{s}_{2}}\hat{\mathbf{Q}}_{2}\le0$ for any $\left(\mathbf{N}_{2}^{T}\mathbf{Q}_{1},-\vec{s}_{2}\right)$-kernel
basis $\hat{\mathbf{Q}}_{2}$ and then apply \prettyref{lem:orderbasisContainsNullspacebasisGeneralized}.
Note that $\rdeg_{-\vec{b}_{2}}\mathbf{N}_{2}^{T}\mathbf{Q}_{1}\le-\vec{s}_{2}$.
Let $\hat{\mathbf{G}}=\mathbf{Q}_{1}\hat{\mathbf{Q}}_{2}$, a $\left(\mathbf{N}^{T},-\vec{s}\right)$-kernel
basis. Note that $\rdeg_{-\vec{s}_{2}}\hat{\mathbf{Q}}_{2}=\rdeg_{-\vec{s}}\mathbf{Q}_{1}\hat{\mathbf{Q}}_{2}=\rdeg_{-\vec{s}}\hat{\mathbf{G}}\le0$. 
\end{proof}
Now that we can correctly compute a $\left(\mathbf{N}^{T},-\vec{s}\right)$-kernel
basis by rows with the help of order basis computation, we need to
look at how to do it efficiently. One major difficulty is that the
order, or equivalently, the $\vec{s}$-row degrees are nonuniform
and can have degree as large as $\xi=\sum\vec{s}$. To overcome this,
we separate the rows of $\mathbf{N}^{T}$ into blocks according to
their $\vec{s}$-row degrees, and then work with these blocks one
by one successively based on \prettyref{thm:continueComputingNullspaceBasisByRows}. 


\begin{algorithm}[t]
\caption{$\mnbr(\mathbf{M},\vec{s})$}
\label{alg:minimalKernelBasisReverse}

\bigskip
\begin{algorithmic}
[1]\REQUIRE{$\mathbf{M}\in\mathbb{K}\left[x\right]^{k\times n}$ such that $\sum\rdeg_{\vec{s}}~\mathbf{M}\le\sum\vec{s}$
and a $\left(\mathbf{M},-\vec{s}\right)$-kernel basis having row degrees
bounded by $\vec{s}$. \\%(equivalently, $-\vec{s}$-column degrees bounded by 0)
}
%\bigskip

\ENSURE{$\mathbf{G}\in\mathbb{K}\left[x\right]^{n\times*}$, a $\left(\mathbf{M},-\vec{s}\right)$-kernel
basis.}
\bigskip

\STATE{\textbf{$\left[\mathbf{M}_{1}^{T},\mathbf{M}_{2}^{T},\cdots,\mathbf{M}_{\log k-1}^{T},\mathbf{M}_{\log k}^{T}\right]:=\mathbf{M}^{T}$},
with $\mathbf{M}_{\log k},\mathbf{M}_{\log k-1},\cdots,\mathbf{M}_{2},\mathbf{M}_{1}$
having $\vec{s}$-row degrees in the range $\left[0,\frac{2\xi}{k} \right],(\frac{2\xi}{k},\frac{4\xi}{k}],...,(\frac{\xi}{4},\frac{\xi}{2}],(\frac{\xi}{2},\xi].$\textbf{ }}

\FOR{$i$ \textbf{from $1$ to $\log k$ }} 

\forbody{\STATE{$\vec{\sigma}_{i}:=\left[\frac{\xi}{2^{i-1}}+1,\dots,\frac{\xi}{2^{i-1}}+1\right]$,
 number of entries matching the row dimension of $\mathbf{M}_{i};$}}

\STATE{$\vec{\sigma}:=\left[\vec{\sigma}_{1},\vec{\sigma}_{2},\dots,\vec{\sigma}_{\log k}\right]$;}

\STATE{$\hat{\mathbf{N}}:=x^{\vec{\sigma}-\vec{b}-1}\mathbf{M};$}

\STATE{$\mathbf{G}_{0}:=I_{n}$; $\tilde{\mathbf{N}}_{0}:=I_{n};$}

\FOR{$i$ \textbf{from $1$ to $\log k$ }} 

\forbody{\STATE{$\vec{s}_{i}:=-\cdeg_{-\vec{s}}\mathbf{G}_{i-1};$ (note $\vec{s}_{1}=\vec{s}$)}

\STATE{$\mathbf{P}_{i}:=\umab\left(\hat{\mathbf{N}}_{i}\tilde{\mathbf{G}}_{i-1},\vec{\sigma}_{i},-\vec{s}_{i}\right)$; }

\STATE{$\left[\mathbf{G}_{i},\mathbf{Q}_{i}\right]:=\mathbf{P}_{i}$, where
$\mathbf{G}_{i}$ is a $\left(\hat{\mathbf{N}}_{i},-\vec{s}_{i}\right)$-kernel
basis;}

\STATE{$\tilde{\mathbf{G}_{i}}:=\tilde{\mathbf{G}}_{i-1}\cdot\mathbf{G}_{i};$ }

}



\RETURN $\tilde{\mathbf{G}_{i}}$ 
\end{algorithmic}
\end{algorithm}



Let $k$ be the column dimension of $\mathbf{N}$. Since $\sum\cdeg_{\vec{s}}\mathbf{N}=\sum\vec{b}\le\sum\vec{s}\le\xi$,
at most $k/c$ columns of $\mathbf{N}$ have $\vec{s}$-column degrees
greater than or equal to $c\xi/k$ for any $c\ge1$. We assume without
loss of generality that the rows of $\mathbf{N}^{T}$ are arranged
in decreasing $\vec{s}$-row degrees. We divide $\mathbf{N}^{T}$
into $\log k$ row blocks according to the $\vec{s}$-row degrees
of its rows, or equivalently, divide $\mathbf{N}$ to blocks of columns
according to the $\vec{s}$-column degrees. Let 
\[
\mathbf{N}=\left[\mathbf{N}_{1},\mathbf{N}_{2},\cdots,\mathbf{N}_{\log k-1},\mathbf{N}_{\log k}\right]
\]
with $\mathbf{N}_{\log k},\mathbf{N}_{\log k-1},\dots,\mathbf{N}_{2},\mathbf{N}_{1}$
having $\vec{s}$-column degrees in the range $\left[0,2\xi/k\right]$,
$(2\xi/k,4\xi/k],$ $(4\xi/k,8\xi/k],\ ...,$ $(\xi/4,\xi/2],$ $(\xi/2,\xi].$
Let $\vec{\sigma}_{i}=\left[\xi/2^{i-1}+1,\dots,\xi/2^{i-1}+1\right]$
with the same dimension as the row dimension of $\mathbf{N}_{i}$.
Let $\vec{\sigma}=\left[\vec{\sigma}_{\log k},\vec{\sigma}_{\log k-1},\dots,\vec{\sigma}_{1}\right]$
be the order in the order basis computation.

To further simply our task, we also make the order of our problem
in each block uniform. Rather than of using $\mathbf{N}^{T}$ as the
input matrix, we use 
\begin{eqnarray*}
\hat{\mathbf{N}} & =\begin{bmatrix}\hat{\mathbf{N}}_{1}\\
\vdots\\
\hat{\mathbf{N}}_{\log k}
\end{bmatrix}= & x^{\vec{\sigma}-\vec{b}-1}\begin{bmatrix}\mathbf{N}_{1}^{T}\\
\vdots\\
\mathbf{N}_{\log k}^{T}
\end{bmatrix}=x^{\vec{\sigma}-\vec{b}-1}\mathbf{N}^{T}
\end{eqnarray*}
 instead, so that a $\left(\hat{\mathbf{N}},\vec{\sigma},-\vec{s}\right)$-basis
is a $\left(\mathbf{N}^{T},\vec{b}+1,-\vec{s}\right)$-basis.

We are now ready to compute a $\left(\mathbf{N}^{T},-\vec{s}\right)$-kernel
basis, which is done by a series of order basis computations that
computes a series of kernel bases as follows.

Let $\vec{s}_{1}=\vec{s}$. First we compute an $\left(\hat{\mathbf{N}}_{1},\vec{\sigma}_{1},-\vec{s}_{1}\right)$-basis
$\mathbf{P}_{1}=\left[\mathbf{G}_{1},\mathbf{Q}_{1}\right]$, where
$\mathbf{G}_{1}$ is a $\left(\hat{\mathbf{N}}_{1},-\vec{s}_{1}\right)$-kernel
basis%
\begin{comment}
 with $\cdeg_{-\vec{s}_{1}}\mathbf{N}_{1}\le0$
\end{comment}
.

Let $\tilde{\mathbf{G}}_{1}=\mathbf{G}_{1}$. Let $\vec{s}_{2}=-\cdeg_{-\vec{s}}\mathbf{G}_{1}$.
\begin{comment}
Note that $-\vec{s}_{1}\le-[\vec{s}_{2},\vec{t}_{2}]\le\left[0,\dots,0,1,\dots1\right]$
component-wise, since $\mathbf{P}_{1}$ has lower order than any $\left(\mathbf{M}^{T},\vec{b}+\left[1,\dots,1\right],-\vec{s}\right)$-basis
$\mathbf{P}$ hence generates $\mathbf{P}$. Therefore, $\cdeg_{-\vec{s}}\mathbf{P}_{1}\le\cdeg_{-\vec{s}}\mathbf{P}\le\left[0,\dots,0,1,\dots1\right]$. 
\end{comment}
{} We then compute an $\left(\hat{\mathbf{N}}_{2}\tilde{\mathbf{G}}_{1},\vec{\sigma}_{2},-\vec{s}_{2}\right)$-basis
$\mathbf{P}_{2}=\left[\mathbf{G}_{2},\mathbf{Q}_{2}\right]$ with
$\vec{s}_{3}=-\cdeg_{-\vec{s}_{2}}\mathbf{G}_{2}$. Let $\tilde{\mathbf{G}}_{2}=\tilde{\mathbf{G}}_{1}\mathbf{G}_{2}$.
 %
\begin{comment}
Let $\mathbf{R}_{1}=\left[\mathbf{N}_{1}\mathbf{Q}_{2},\mathbf{Q}_{1}\right]$
and $\mathbf{R}_{1}^{r}=\revCol\left(\mathbf{R}_{1},-\vec{s},\cdeg_{-\vec{s}}\mathbf{R}_{1}\right)$.
Then from \prettyref{lem:unimodularComputationByRows} we know $\left[\mathbf{F}^{T},\mathbf{R}_{1}^{r}\right]$
is a unimodular matrix.
\end{comment}


Continuing this process, at step $i$ we compute an $\left(\hat{\mathbf{N}}_{i}\tilde{\mathbf{G}}_{i-1},\vec{\sigma}_{i},-\vec{s}_{i}\right)$-basis
$\mathbf{P}_{i}=\left[\mathbf{G}_{i},\mathbf{Q}_{i}\right]$. Let
$\tilde{\mathbf{G}}_{i}=\prod_{j=1}^{i}\mathbf{G}_{i}=\tilde{\mathbf{G}}_{i-1}\mathbf{G}_{i}$.
Note that $\tilde{\mathbf{G}}_{\log k}$ is a $\left(\mathbf{N}^{T},-\vec{s}\right)$-kernel
basis. 

This process of computing a $\left(\mathbf{N}^{T},-\vec{s}\right)$-kernel
basis gives \prettyref{alg:minimalNullspaceBasisReverse}.

Now let us check the cost of this algorithm. The cost is dominated
by the order basis computation and the multiplications $\hat{\mathbf{N}}_{i}\tilde{\mathbf{G}}_{i-1}$
and $\tilde{\mathbf{G}}_{i-1}\mathbf{G}_{i}$.
\begin{lem}
An $\left(\hat{\mathbf{N}}_{i}\tilde{\mathbf{G}}_{i-1},\vec{\sigma}_{i},-\vec{s}_{i}\right)$-basis
can be computed at a cost of $O^{\sim}\left(n^{\omega}d\right)$,
where $d=\xi/n$.\end{lem}
\begin{proof}
Note that $\mathbf{N}_{i}$ has less than $2^{i}$ columns. Otherwise,
$\sum\cdeg_{\vec{s}}\mathbf{N}_{i}>2^{i}\xi/2^{i}=\xi$, contradicting
with $\sum\cdeg_{\vec{s}}\mathbf{N}=\sum\vec{b}\le\sum\vec{s}\le\xi$.
It follows that $\hat{\mathbf{N}}_{i}$, and therefore $\hat{\mathbf{N}}_{i}\tilde{\mathbf{G}}_{i-1}$,
also have less than $2^{i}$ rows. We also have $\vec{\sigma}_{i}=\left[\xi/2^{i-1}+1,\dots,\xi/2^{i-1}+1\right]=\Theta\left(\xi/2^{i}\right)$.
Therefore, Algorithm 2 from \citep{za2012} for order basis computation
with unbalanced shift can be used with a cost of $O^{\sim}\left(n^{\omega}d\right)$.\end{proof}
\begin{lem}
The multiplications $\hat{\mathbf{N}}_{i}\tilde{\mathbf{G}}_{i-1}$
can be done with a cost of $O^{\sim}\left(n^{\omega}d\right)$.\end{lem}
\begin{proof}
The dimension of $\hat{\mathbf{N}}_{i}$ is bounded by $2^{i-1}\times n$
and $\sum\rdeg_{\vec{s}}\hat{\mathbf{N}}_{i}\le2^{i-1}\cdot\xi/2^{i-1}=\xi$.
We also have $\cdeg_{-\vec{s}}\tilde{\mathbf{G}}_{i-1}\le0$, or equivalently,
$\rdeg\tilde{\mathbf{G}}_{i-1}\le\vec{s}$. We can now use \prettyref{thm:multiplyUnbalancedMatrices}
to multiply $\tilde{\mathbf{G}}_{i-1}^{T}$ and $\hat{\mathbf{N}}_{i}^{T}$
with a cost of $O^{\sim}\left(n^{\omega-1}\xi\right)=O^{\sim}\left(n^{\omega}d\right)$.\end{proof}
\begin{lem}
The multiplication $\tilde{\mathbf{G}}_{i-1}\mathbf{G}_{i}$ can be
done with a cost of $O^{\sim}\left(n^{\omega}d\right)$.\end{lem}
\begin{proof}
We know $\cdeg_{-\vec{s}}\tilde{\mathbf{G}}_{i-1}=-\vec{s}_{i}$,
and $\cdeg_{-\vec{s}_{i}}\mathbf{G}_{i}=-\vec{s}_{i+1}\le0.$ In other
words, $\rdeg\mathbf{G}_{i}\le\vec{s}_{i}$, and $\rdeg_{\vec{s}_{i}}\tilde{\mathbf{G}}_{i-1}\le\vec{s}$,
hence we can again use \prettyref{thm:multiplyUnbalancedMatrices}
to multiply $\mathbf{G}_{i}^{T}$ and $\tilde{\mathbf{G}}_{i-1}^{T}$
with a cost of $O^{\sim}\left(n^{\omega}d\right)$.\end{proof}
\begin{thm}
A right factor $\mathbf{G}$ satisfying $\mathbf{TG}=\mathbf{F}$
for a column basis $\mathbf{T}$ can be computed with a cost of $O^{\sim}\left(n^{\omega-1}\xi\right)=O^{\sim}\left(n^{\omega}d\right)$.
\end{thm}

\section{Computing a Column Basis}

With a right factor $\mathbf{G}$ of $\mathbf{F}$ computed, we are
now ready to compute a column basis $\mathbf{T}$, where $\mathbf{F}=\mathbf{T}\mathbf{G}$.
To do so efficiently, the degree of $\mathbf{T}$ cannot be too big,
which is indeed the case as shown by the following lemmas.
\begin{lem}
\label{lem:colBasisdegreeBoundByRdegOfRightFactor}The column degrees
of $\mathbf{T}$ are bounded by the corresponding entries of $\vec{t}=-\rdeg_{-\vec{s}}\mathbf{G}$.\end{lem}
\begin{proof}
Since $\mathbf{G}$ is $-\vec{s}$-row reduced, and $\rdeg_{-\vec{s}}\mathbf{F}\le0$,
by \prettyref{lem:predictableDegree} $\rdeg_{-\vec{t}}\mathbf{T}\le0$,
or equivalently, $\mathbf{T}$ has column degrees bounded by $\vec{t}$.\end{proof}
\begin{lem}
\label{lem:colBasisDegreeBoundByInputDegrees}Let $\vec{t}=-\rdeg_{-\vec{s}}\mathbf{G}$,
a vector with $r$ entries and bounds $\cdeg\mathbf{T}$ from \prettyref{lem:colBasisdegreeBoundByRdegOfRightFactor}.
Let $\vec{s}'$ be the list of the $r$ largest entries of $\vec{s}$.
Then $\vec{t}\le\vec{s}'$.\end{lem}
\begin{proof}
Let $\mathbf{G}'$ be the $-\vec{s}$-row Popov form of $\mathbf{G}$,
and the square matrix $\mathbf{G}"$ consists of only the columns
of $\mathbf{G}'$ that contains pivot entries, and has the rows permuted
so the pivots are in the diagonal. Let $\vec{s}"$ be the list of
the entries in $\vec{s}$ that correspond to the columns of $\mathbf{G}"$
in $\mathbf{G}'$. Note that $\rdeg_{-\vec{s}"}\mathbf{G}"=-\vec{t}"$
is just a permutation of $-\vec{t}$ with the same entries. By the
definition of shifted row degree, $-\vec{t}"$ is the sum of $-\vec{s}"$
and the list of the diagonal pivot degrees, which are nonnegative.
Therefore, $-\vec{t}"\ge-\vec{s}"$. The result then follows as $\vec{t}$
is a permutation of $\vec{t}"$ and $\vec{s}'$ has the largest entries
of $\vec{s}$.
\end{proof}
With the bound on the column degrees of $\mathbf{T}$ determined,
we are now ready to compute $\mathbf{T}$. This is done again using
an order basis computation.
\begin{lem}
Let $\vec{t}'=\left[0,\dots,0,\vec{t}\right]\in\mathbb{Z}^{m+r}$.
Any $\left(\left[\mathbf{F}^{T},\mathbf{G}^{T}\right],-\vec{t}'\right)$-kernel
basis has the form $\begin{bmatrix}V\\
\bar{\mathbf{T}}
\end{bmatrix}$, where $V\in\mathbb{K}^{m\times m}$ is a unimodular matrix and $\left(\bar{\mathbf{T}}V^{-1}\right)^{T}$
is a column basis of $\mathbf{F}$.\end{lem}
\begin{proof}
First, the matrix $\begin{bmatrix}I\\
\mathbf{T}^{T}
\end{bmatrix}$ is a kernel basis of $\left[\mathbf{F}^{T},\mathbf{G}^{T}\right]$
and is unimodularly equivalent to any other kernel basis. Hence we
must have $\begin{bmatrix}V\\
\bar{\mathbf{T}}
\end{bmatrix}=\begin{bmatrix}I\\
\mathbf{T}^{T}
\end{bmatrix}V$, which gives $\mathbf{T}=\left(\bar{\mathbf{T}}V^{-1}\right)^{T}$.
Also note that the $-\vec{t}'$-minimality forces the unimodular matrix
$V$ in any $\left(\left[\mathbf{F}^{T},\mathbf{G}^{T}\right],-\vec{t}'\right)$-kernel
basis to be the same degree as $I$.
\end{proof}
To compute a $\left(\left[\mathbf{F}^{T},\mathbf{G}^{T}\right],-\vec{t}'\right)$-kernel
basis, we can again use order basis computation.
\begin{lem}
Any $\left(\left[\mathbf{F}^{T},\mathbf{G}^{T}\right],\vec{s}+\left[1,\dots,1\right],-\vec{t}'\right)$-basis
contain a $\left(\left[\mathbf{F}^{T},\mathbf{G}^{T}\right],-\vec{t}'\right)$-kernel
basis whose $-\vec{t}'$-row degrees are bounded by 0.\end{lem}
\begin{proof}
As before, \prettyref{lem:orderbasisContainsNullspacebasisGeneralized}
can be used here. We just need to show that a $\left(\left[\mathbf{F}^{T},\mathbf{G}^{T}\right],-\vec{t}'\right)$-kernel
basis has $-\vec{t}'$-row degrees no more than 0, which is true since
$\rdeg_{-\vec{t}'}\begin{bmatrix}I\\
\mathbf{T}^{T}
\end{bmatrix}\le0$.
\end{proof}
Now to compute a $\left(\left[\mathbf{F}^{T},\mathbf{G}^{T}\right],-\vec{t}'\right)$-kernel
basis efficiently, notice we have the same type of problem as in \prettyref{sec:computeRightFactor},
and so \prettyref{alg:minimalNullspaceBasisReverse} works here as
well. With a $\left(\left[\mathbf{F}^{T},\mathbf{G}^{T}\right],-\vec{t}'\right)$-kernel
basis $\begin{bmatrix}V\\
\bar{\mathbf{T}}
\end{bmatrix}$ computed, a column basis can now be easily computed by $\mathbf{T}=\left(\bar{\mathbf{T}}V^{-1}\right)^{T}$.

We now have a complete algorithm for computing a column basis, given
in \prettyref{alg:colBasis}.

\input{algorithmColumnBasis.tex}
\begin{thm}
\label{thm:columnBasisCost1}A column basis $\mathbf{T}$ of $\mathbf{F}$
can be computed with a cost of $O^{\sim}\left(n^{\omega-1}\xi\right)$.
\end{thm}
\begin{comment}
This can be done by computing a left $\left[0,\dots,0,d,\dots,d\right]$-minimal
kernel basis $\left[\mathbf{T}',V\right]$ of $\left[\mathbf{G}^{T},\mathbf{F}^{T}\right]^{T}$,
where $d$ is the degree of $\mathbf{F},$ $V$ is a unimodular matrix
and $\mathbf{T}'$ is a $m\times m$ matrix. Note that $\left[\mathbf{T}',V\right]$
has $m$ rows since the rank of $\left[\mathbf{G}^{T},\mathbf{F}^{T}\right]^{T}$
is $m$. Also note that since $\left[\mathbf{T},I\right]$ is a left
kernel basis with $\left[0,\dots,0,d,\dots,d\right]$-row degrees
bounded by $d$, the $\left[0,\dots,0,d,\dots,d\right]$-minimal kernel
basis $\left[\mathbf{T}',V\right]$ must also has its $\left[0,\dots,0,d,\dots,d\right]$-row
degrees bounded by $d$, hence the degree of $V$ must be 0. We can
then easily compute $\mathbf{T}=\mathbf{T}'V^{-1}$. 
\end{comment}



\section{\label{sec:successiveColBasisComputation}A Simple Improvement}

When the input matrix $\mathbf{F}$ has column dimension much larger
$n$ than the row dimension $m$, we can separate $\mathbf{F}=\left[\mathbf{F}_{1},\mathbf{F}_{2},\dots,\mathbf{F}_{n/m}\right]$
to $n/m$ blocks, each with dimension $m\times m$, assuming without
loss of generality $n$ is a multiple of $m$, and the columns are
arranged in increasing degrees. We then do a series of column basis
computations. First we compute a column basis $\mathbf{T}_{1}$ of
$\left[\mathbf{F}_{1},\mathbf{F}_{2}\right]$. Then compute a column
basis $\mathbf{T}_{2}$ of $\left[\mathbf{T}_{1},\mathbf{F}_{3}\right]$.
Repeating this process, at step $i$, we compute a column basis $\mathbf{T}_{i}$
of $\left[\mathbf{T}_{i-1},\mathbf{F}_{i+1}\right]$, until $i=n/m-1$,
when a column basis of $\mathbf{F}$ is computed.
\begin{lem}
At step $i$, computing a column basis $\mathbf{T}_{i}$ of $\left[\mathbf{T}_{i-1},\mathbf{F}_{i+1}\right]$
can be done with a cost of $O^{\sim}\left(m^{\omega-1}\left(\sum\cdeg\mathbf{F}_{i}+\sum\cdeg\mathbf{F}_{i+1}\right)\right)$
field operations.\end{lem}
\begin{proof}
From \prettyref{lem:colBasisDegreeBoundByInputDegrees}, the column
basis $\mathbf{T}_{i-1}$ of $\left[\mathbf{F}_{1},\dots,\mathbf{F}_{i}\right]$
has column degrees bounded by the largest column degrees of $\mathbf{F}_{i}$,
hence $\sum\cdeg\mathbf{T}_{i-1}\le\sum\cdeg\mathbf{F}_{i}$. The
lemma then follows by combining this with the result that a column
basis $\mathbf{T}_{i}$ of $\left[\mathbf{T}_{i-1},\mathbf{F}_{i+1}\right]$
can be computed with a cost of $O^{\sim}\left(m^{\omega-1}\left(\sum\cdeg\mathbf{T}_{i-1}+\sum\cdeg\mathbf{F}_{i+1}\right)\right)$
from \prettyref{thm:columnBasisCost1}.\end{proof}
\begin{thm}
\label{thm:finalCollBasisCost}A column basis of $\mathbf{F}$ can
be computed with a cost of $O^{\sim}\left(m^{\omega-1}\xi\right)$.\end{thm}
\begin{proof}
Summing up the cost of all the column basis computations, 
\begin{eqnarray*}
 &  & \sum_{i=1}^{n/m-1}O^{\sim}\left(m^{\omega-1}\left(\sum\cdeg\mathbf{F}_{i}+\sum\cdeg\mathbf{F}_{i+1}\right)\right)\\
 & \subset & O^{\sim}\left(m^{\omega-1}\left(2\sum\cdeg\mathbf{F}\right)\right)\\
 & = & O^{\sim}\left(m^{\omega-1}\xi\right)
\end{eqnarray*}
\end{proof}
\begin{rem}
In this section, the computational efficiency is improved by reducing
the original problem to about $n/m$ subproblems whose column dimensions
are close to the row dimension $m$. This is done by successive column
basis computations. Note that we can also reduce the column dimension
by using successive order basis computations, and only do a column
basis computation at the very last step. The computational complexity
of using order basis computation to reduce the column dimension would
remain the same, but in practice it maybe more efficient since order
basis computations are simpler.
\end{rem}

\section{Conclusion\label{sec:Future-Research}}

In this paper we have presented a fast, deterministic procedure for
computing a column basis of a polynomial matrix. 

\bibliographystyle{plainnat}
\bibliography{paper}

\end{document}
