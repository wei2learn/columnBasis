%% LyX 2.0.5.1 created this file.  For more info, see http://www.lyx.org/.
%% Do not edit unless you really know what you are doing.
\documentclass[twoside]{sig-alternate}
\usepackage[T1]{fontenc}
\usepackage[latin9]{inputenc}
\usepackage{color}
\usepackage{verbatim}
\usepackage{prettyref}
\usepackage{float}
\usepackage{amsmath}
\usepackage{amssymb}

\makeatletter

%%%%%%%%%%%%%%%%%%%%%%%%%%%%%% LyX specific LaTeX commands.
\floatstyle{ruled}
\newfloat{algorithm}{tbp}{loa}
\providecommand{\algorithmname}{Algorithm}
\floatname{algorithm}{\protect\algorithmname}

%%%%%%%%%%%%%%%%%%%%%%%%%%%%%% Textclass specific LaTeX commands.
\floatstyle{ruled}
\newfloat{algorithm}{tbp}{loa}
\floatname{algorithm}{Algorithm}
\usepackage{algorithmic}
\newcommand{\forbody}[1]{ #1 \ENDFOR }
\newcommand{\ifbody}[1]{ #1  \ENDIF}
\newcommand{\whilebody}[1]{ #1  \ENDWHILE}
\renewcommand{\algorithmicprint}{\textbf{draw}}
\renewcommand{\algorithmicrequire}{\textbf{Input:}}
\renewcommand{\algorithmicensure}{\textbf{Output:}}

 \newtheorem{thm}{Theorem}[section]
 \newtheorem{lem}[thm]{Lemma}
 \newtheorem{defn}[thm]{Definition}
 \newtheorem{rem}[thm]{Remark}
 \newtheorem{exmp}[thm]{Example}

\@ifundefined{date}{}{\date{}}
%%%%%%%%%%%%%%%%%%%%%%%%%%%%%% User specified LaTeX commands.
%\usepackage{yjsco}\journal{JournalofSymbolicComputation}

\renewcommand{\algorithmicrequire}{\textbf{Input:}}\renewcommand{\algorithmicensure}{\textbf{Output:}}
%\renewcommand{\algorithmicensure}{\textbf{if}}\renewcommand{\algorithmicensure}{\textbf{Uses:}}

%\def\diag{\mbox{diag}}\def\cdeg{\qopname\relax n{cdeg}}
\def\MM{\qopname\relax n{MM}}\def\M{\qopname\relax n{M}}
%\def\ord{\qopname\relax n{ord}}

\def\StorjohannTransform{\qopname\relax n{StorjohannTransform}}\def\TransformUnbalanced{\qopname\relax n{TransformUnbalanced}}\def\rowDimension{\qopname\relax n{rowDimension}}\def\columnDimension{\qopname\relax n{columnDimension}}\DeclareMathOperator{\re}{rem}\DeclareMathOperator{\coeff}{coeff}\DeclareMathOperator{\lcoeff}{lcoeff}\def\mab{\qopname\relax n{orderBasis}}\def\mmab{\qopname\relax n{FastBasis}}\def\umab{\qopname\relax n{UnbalancedFastBasis}}\newcommand{\bb}{\\}
\def\mnb{\qopname\relax n{MinimalNullspaceBasis ~ }}
\DeclareMathOperator{\mnbr}{minimaKernelBasisReversed}
%\def\rdeg{\qopname\relax n{rdeg}}
\DeclareMathOperator{\colBasis}{colBasis}


\makeatother
\newcommand{\arne}[1]{{\color{blue}\it {\bf Arne:} #1 }}
\newcommand{\wei}[1]{{\color{red}\it {\bf Wei:} #1}}
\newcommand{\george}[1]{{\color{green}\it {\bf George:} #1}}
\def\newblock{\hskip .11em plus .33em minus .07em}


\newcommand{\Z}{\mathbb{Z}}
\newcommand{\Q}{\mathbb{Q}}
\newcommand{\N}{\mathbb{N}}
\newcommand{\K}{ {\rm K}}
\newcommand{\revCol}{ {\rm revCol}}
\newcommand{\bigO}{\mathcal{O}}
\newcommand{\tbigO}{\widetilde{\mathcal{O}}}
\newcommand{\cL}{\mathcal{L}}
\newcommand{\ocL}{\overline{\mathcal{L}}}
\newcommand{\tcL}{\widetilde{\mathcal{L}}}
\newcommand{\cA}{\mathcal{A}}
\newcommand{\oA}{\overline{A}}
\newcommand{\GL}{{\rm GL}\,}
\newcommand{\rank}{{\rm rank}\,}
\newcommand{\cdeg}{{\rm cdeg}\,}
\newcommand{\rdeg}{{\rm rdeg}\,}
\newcommand{\diag}{{\rm diag}\,}
\newcommand{\val}{{\rm val}\,}
\newcommand{\ord}{{\rm ord}\,}
\newcommand{\abs}[1]{\lvert#1\rvert}


\newrefformat{eq}{\textup{(\ref{#1})}}
\newrefformat{lem}{Lemma \ref{#1}}
\newrefformat{cla}{Claim \ref{#1}}
\newrefformat{thm}{Theorem \ref{#1}}
\newrefformat{cha}{Chapter \ref{#1}}
\newrefformat{sec}{Section \ref{#1}}
\newrefformat{rem}{Remark \ref{#1}}
\newrefformat{fac}{Fact \ref{#1}}
\newrefformat{sub}{Subsection \ref{#1}}
\newrefformat{cor}{Corollary \ref{#1}}
\newrefformat{cond}{Condition \ref{#1}}
\newrefformat{con}{Conjecture \ref{#1}}
\newrefformat{def}{Definition \ref{#1}}
\newrefformat{pro}{Proposition \ref{#1}}
\newrefformat{alg}{Algorithm \ref{#1}}
\newrefformat{exm}{Example \ref{#1}}
\newrefformat{line}{line \ref{#1}}
\newrefformat{tab}{Table \ref{#1} on page \pageref{#1}}
\newrefformat{fig}{Figure \ref{#1} on page \pageref{#1}}

\makeatother

\begin{document}
% --- Author Metadata here ---


%\conferenceinfo{ISSAC'09,} {July 28--31, 2009, Seoul, Republic of Korea.} 
%\CopyrightYear{2009}
%\crdata{978-1-60558-609-0/09/07} 

\numberofauthors{1}


\author{\alignauthor Wei Zhou and George Labahn\\
 \affaddr{Cheriton School of Computer Science}\\
 \affaddr{University of Waterloo}, \\
 \affaddr{Waterloo, Ontario, Canada}\\
 \email{\{w2zhou,glabahn\}@uwaterloo.ca} }


\date{{\normalsize{\today \quad{}:: \timeofday}}}


\title{Computing Column Bases}
\maketitle
\begin{abstract}
Given a matrix of univariate polynomials over a field $\mathbb{K}$,
its columns generate a $\mathbb{K}\left[x\right]$-module consisting
of all $\mathbb{K}\left[x\right]$-linear combination of these columns.
We call any basis of this module a column basis of the given matrix.
In this paper we present a deterministic algorithm for the computation
of a column basis of an $m\times n$ input matrix with $m\le n$.
If $s$ is the average column degree of the input matrix, this algorithm
computes a column basis with a cost of $O^{\sim}\left(nm^{\omega-1}s\right)$
field operations in $\mathbb{K}$. Here the soft-$O$ notation is
Big-$O$ with log factors removed while $\omega$ is the exponent
of matrix multiplication. Note that the average column degree $s$
is bounded by the commonly used matrix degree that is also the maximum
column degree of the input matrix.
\end{abstract}
% A category with the (minimum) three required fields
%\category{I.1.2}{Symbolic and Algebraic Manipulation}{Algorithms}%[Algebraic algorithms]
%A category including the fourth, optional field follows...
%\category{F.2.2}{Analysis of Algorithms and }{Nonnumerical Algorithms and Problems}%[Computations on discrete structures]

%\terms{Algorithms, Theory}

%\keywords{Order basis, Complexity}

\vspace{1mm} \noindent {\bf Categories and Subject Descriptors:} I.1.2 {[{Symbolic and Algebraic Manipulation}]}: {Algorithms}; F.2.2 {[{Analysis of Algorithms and Problem Complexity}]}: {Nonnumerical Algorithms and Problems}

\vspace{1mm} \noindent {\bf General Terms:} Algorithms, Theory

\vspace{1mm} \noindent {\bf Keywords:} Nullspace basis, Complexity


\section{\label{sec:Matrix-GCD}Introduction}

In this paper, we consider the problem of efficiently computing a
column basis of a polynomial matrix $\mathbf{F}\in\mathbb{K}\left[x\right]^{m\times n}$
with $n\ge m$. A column basis of $\mathbf{F}$ is a basis for the
$\mathbb{K}\left[x\right]$-module 
\[
\left\{ \mathbf{F}\mathbf{p}~|~\mathbf{p}\in\mathbb{K}\left[x\right]^{n}~\right\} ~.
\]
Such a basis can be represented as a full rank matrix 
$\mathbf{T}\in\mathbb{K}\left[x\right]^{m\times r}$
whose columns are the basis elements. A column basis is not
unique and indeed any column basis right multiplied by a unimodular
polynomial matrix gives another column basis. As a result, a column basis can
have arbitrarily high degree. In this paper, the computed column basis 
has column degrees bounded by the largest column degrees of the input
matrix.

Column bases are fundamental constructions in polynomial matrix algebra.
As an example, when the row dimension is one (i.e. $m=1$), then finding a column basis
coincides with finding a greatest common divisor (GCD) of all the
polynomials in the matrix. Similarly, the nonzero columns of column
reduced forms, Popov normal forms, and Hermite normal forms are all
column bases satisfying additional degree constraints. A column reduced
form gives a special column basis whose column degrees are the smallest
possible, while Popov and Hermite forms are special column reduced
or shifted column reduced forms satisfying additional conditions that
make them unique. Efficient column basis computation is thus useful
for fast computation for such core procedures as determining matrix
GCDs \cite{BL2000}, column reduced forms \cite{BVP:1988} and Popov
forms \cite{villard96} of $\mathbf{F}$ with any dimension and rank.
Column basis computation %in this paper 
also provides a deterministic
alternative to randomized lattice compression \cite{li:2006,storjohann-villard:2005}.

While column bases are produced by column reduced, Popov and Hermite forms and considerable
work has been done on computing such forms, for example
\cite{bcl:2006,beelen:1988,Giorgi2003,GSSV2012,sarkar2011,SS2011}. However
most of these existing algorithms require that the  input matrices be square 
nonsingular and so start with existing column bases. It is however pointed
out in \cite{sarkar2011,SS2011} that randomization can be used to
relax the square nonsingular requirement.

To compute a column basis, we know from \cite{BL1997} that any matrix
polynomial $\mathbf{F}\in\mathbb{K}\left[x\right]^{m\times n}$ can
be unimodularly transformed to a column basis by repeatedly working
with the leading column coefficient matrices. However this method
of computing a column basis can be expensive. Indeed one needs to
work with up to $\sum\vec{s}$ such coefficient matrices, which could
involve up to $\sum\vec{s}$ polynomial matrix multiplications.

In this paper we give a fast, deterministic algorithm for the computation
of a column basis for $\mathbf{F}$ having complexity $O^{\sim}\left(nm^{\omega-1}s\right)$
field operations in $\mathbb{K}$ with $s$ being the average average
column degree of $\mathbf{F}$. Here the soft-$O$ notation is Big-$O$
with log factors removed while $\omega$ is the exponent of matrix
multiplication. Our algorithm works for both rectangular and non-full
column rank matrices. Our method depends on kernel computation
of $\mathbf{F}$ along with finding a factorization of the input matrix
polynomial into a column basis and a left kernel of the right kernel.
Finding the right and left kernel then makes use of the fast kernel
and order basis algorithms from \cite{za2012} and \cite{za2009},
respectively.

The remainder of this paper is as follows. Basic definitions and preliminary
results on both kernel and order bases are given in the next section.
Section 3 provides the matrix factorization form of our input polynomial
matrix that forms the core of our procedure, with a column basis being
the left factor, and the right factor is a left kernel basis of a
right kernel basis of the input matrix. Section 4 provides an algorithm
for fast computation of a left kernel making use of order bases computation
with unbalanced shift. The column basis algorithm is given in Section
5 with the following section giving details on how the methods can
be improved when the number of columns is significantly larger than
the number of rows.  The paper ends with a conclusion along with
topics for future research.



\section{Preliminaries}

In this paper computational cost is analyzed by bounding the number
of arithmetic operations in the coefficient field $\mathbb{K}$ on
an algebraic random access machine. We assume the cost of multiplying
two polynomial matrices with dimension $n$ and degree $d$ is $O^{\sim}(n^{\omega}d)$
field operations, where the multiplication exponent $\omega$ is assumed
to satisfy $2<\omega\le3$. We refer to the book by \cite{vonzurgathen}
for more details and references about the cost of polynomial multiplication
and matrix multiplication.

In this section we first describe the notations used in this paper,
and then give the basic definitions and properties of {\em shifted
degree}, {\em order basis} and {\em kernel basis} for a matrix
of polynomials. These will be the building blocks used in our algorithm.


\subsection{Notations}

For convenience we adopt the following notations in this paper. 
\begin{description}
\item [{Comparing~Unordered~Lists}] For two lists $\vec{a}\in\mathbb{Z}^{n}$
and $\vec{b}\in\mathbb{Z}^{n}$, let $\bar{a}=\left[\bar{a}_{1},\dots,\bar{a}_{n}\right]\in\mathbb{Z}^{n}$
and $\bar{b}=\left[\bar{b}_{1},\dots,\bar{b}_{n}\right]\in\mathbb{Z}^{n}$
be the lists consists of the entries of $\vec{a}$ and $\vec{b}$
but sorted in increasing order. 
\[
\begin{cases}
\vec{a}\ge\vec{b} & \mbox{if }\bar{a}_{i}\ge\bar{b}_{i}\mbox{ for all }i\in\left[1,\dots n\right]\\
\vec{a}\le\vec{b} & \mbox{if }\bar{a}_{i}\le\bar{b}_{i}\mbox{ for all }i\in\left[1,\dots n\right]\\
\vec{a}>\vec{b} & \mbox{if }\vec{a}\ge\vec{b}\mbox{ and }\bar{a}_{j}>\bar{b}_{j}\mbox{ for at least one }j\in\left[1,\dots n\right]\\
\vec{a}<\vec{b} & \mbox{if }\vec{a}\le\vec{b}\mbox{ and }\bar{a}_{j}<\bar{b}_{j}\mbox{ for at least one }j\in\left[1,\dots n\right].
\end{cases}
\]

\item [{Summation~Notation}] For a list $\vec{a}=\left[a_{1},\dots,a_{n}\right]\in\mathbb{Z}^{n}$,
we write $\sum\vec{a}$ without index to denote the summation of all
entries in $\vec{a}$. 
%\item [{}]~
\item [{Uniformly~Shift~a~List}] For a list 
$\vec{a}=\left[a_{1},\dots,a_{n}\right]\in\mathbb{Z}^{n}$
and $c\in\mathbb{Z}$, we write $\vec{a}+c$ to denote $\vec{a}+\left[c,\dots,c\right]=\left[a_{1}+c,\dots,a_{n}+c\right]$, with subtraction handled similarly.
\item [{Compare~a~List~with~a~Integer}] For %a list 
$\vec{a}=\left[a_{1},\dots,a_{n}\right]\in\mathbb{Z}^{n}$
and $c\in\mathbb{Z}$, we write $\vec{a}<c$ to denote $\vec{a}<\left[c,\dots,c\right]$,
and similarly for $>,\le,\ge,=$.
\end{description}

\subsection{Shifted Degrees}

Our methods depend extensively on the concept of {\em shifted}
degrees of polynomial matrices \cite{BLV:1999}. For a column vector
$\mathbf{p}=\left[p_{1},\dots,p_{n}\right]^{T}$ of univariate polynomials
over a field $\mathbb{K}$, its column degree, denoted by $\cdeg\mathbf{p}$,
is the maximum of the degrees of the entries of $\mathbf{p}$, that
is, 
\[
\cdeg~\mathbf{p}=\max_{1\le i\le n}\deg p_{i}.
\]
The \emph{shifted column degree} generalizes this standard column
degree by taking the maximum after shifting the degrees by a given
integer vector that is known as a \emph{shift}. More specifically,
the shifted column degree of $\mathbf{p}$ with respect to a shift
$\vec{s}=\left[s_{1},\dots,s_{n}\right]\in\mathbb{Z}^{n}$, or the
\emph{$\vec{s}$-column degree} of $\mathbf{p}$ is 
\[
\cdeg_{\vec{s}}~\mathbf{p}=\max_{1\le i\le n}[\deg p_{i}+s_{i}]=\deg(x^{\vec{s}}\cdot\mathbf{p}),
\]
where 
\[
x^{\vec{s}}=\diag\left(x^{s_{1}},x^{s_{2}},\dots,x^{s_{n}}\right) ~.
\]
For a matrix $\mathbf{P}$, we use $\cdeg\mathbf{P}$ and $\cdeg_{\vec{s}}\mathbf{P}$
to denote respectively the list of its column degrees and the list
of its shifted $\vec{s}$-column degrees. When $\vec{s}=\left[0,\dots,0\right]$,
the shifted column degree specializes to the standard column degree.
The shifted row degree of a row vector \textbf{$\mathbf{q}=\left[q_{1},\dots,q_{n}\right]$}
is defined similarly as 
\[
\rdeg_{\vec{s}}\mathbf{q}=\max_{1\le i\le n}[\deg q_{i}+s_{i}]=\deg(\mathbf{q}\cdot x^{\vec{s}}).
\]


Shifted degrees have been used previously in polynomial matrix
computations and in generalizations of some matrix normal forms \cite{BLV:jsc06}.
The shifted column degree is equivalent to the notion of \emph{defect}
commonly used in the literature.

The usefulness of the shifted degrees can be seen from their applications
in polynomial matrix computation problems \cite{ZL2012,za2012}. One
of its uses is illustrated by the following lemma from \cite[Chapter 2]{zhou:phd2012},
which can be viewed as a stronger version of the predictable-degree
property \cite{kailath:1980}. 
\begin{lem}
\label{lem:predictableDegree}Let $\mathbf{A}\in\mathbb{K}\left[x\right]^{m\times n}$
be a $\vec{u}$-column reduced matrix with no zero columns and with
$\cdeg_{\vec{u}}\mathbf{A}=\vec{v}$. Then a matrix $\mathbf{B}\in\mathbb{K}\left[x\right]^{n\times k}$
has $\vec{v}$-column degrees $\cdeg_{\vec{v}}\mathbf{B}=\vec{w}$
if and only if $\cdeg_{\vec{u}}\left(\mathbf{A}\mathbf{B}\right)=\vec{w}$. 
\end{lem}
The following lemma from \cite[Chapter 2]{zhou:phd2012} describes
a relationship between shifted column degrees and shifted row degrees.
\begin{lem}
\label{lem:columnDegreesRowDegreesSymmetry}A matrix $\mathbf{A}\in\mathbb{K}\left[x\right]^{m\times n}$
has $\vec{u}$-column degrees bounded by $\vec{v}$ if and only if
its $-\vec{v}$-row degrees are bounded by $-\vec{u}$. 
\end{lem}
Another essential fact needed in our algorithm, also based on the
use of the shifted degrees, is the efficient multiplication of matrices
with unbalanced degrees \cite[Theorem 3.7]{za2012}. 
\begin{thm}
\label{thm:multiplyUnbalancedMatrices} Let $\mathbf{A}\in\mathbb{K}\left[x\right]^{m\times n}$
with $m\le n$, $\vec{s}\in\mathbb{Z}^{n}$ a shift with entries bounding
the column degrees of $\mathbf{A}$ and $\xi$, a bound on the sum
of the entries of $\vec{s}$. Let $\mathbf{B}\in\mathbb{K}\left[x\right]^{n\times k}$
with $k\in O\left(m\right)$ and the sum $\theta$ of its $\vec{s}$-column
degrees satisfying $\theta\in O\left(\xi\right)$. Then we can multiply
$\mathbf{A}$ and $\mathbf{B}$ with a cost of $O^{\sim}(n^{2}m^{\omega-2}s)\subset O^{\sim}(n^{\omega}s)$,
where $s=\xi/n$ is the average of the entries of $\vec{s}$. 
\end{thm}

\subsection{Order Basis}

Let $\mathbb{K}$ be a field, $\mathbf{F}\in\mathbb{K}\left[\left[x\right]\right]^{m\times n}$
a matrix of power series and $\vec{\sigma}=\left[\sigma_{1},\dots,\sigma_{m}\right]$
a vector of non-negative integers. 
\begin{defn}
A vector of polynomials $\mathbf{p}\in\mathbb{K}\left[x\right]^{n\times1}$
has \emph{order} $\left(\mathbf{F},\vec{\sigma}\right)$ (or \emph{order}
$\vec{\sigma}$ with respect to $\mathbf{F}$) if $\mathbf{F}\cdot\mathbf{p}\equiv\mathbf{0}\mod x^{\vec{\sigma}}$,
that is, 
\[
\mathbf{F}\cdot\mathbf{p}=x^{\vec{\sigma}}\mathbf{r}
\]
for some $\mathbf{r}\in\mathbb{K}\left[\left[x\right]\right]^{m\times1}$.
If $\vec{\sigma}=\left[\sigma,\dots,\sigma\right]$ has entries uniformly
equal to $\sigma$, then we say that $\mathbf{p}$ has order $\left(\mathbf{F},\sigma\right).$
The set of all order $\left(\mathbf{F},\vec{\sigma}\right)$ vectors
is a free $\mathbb{K}\left[x\right]$-module denoted by $\left\langle \left(\mathbf{F},\vec{\sigma}\right)\right\rangle $. 
\end{defn}
An order basis for $\mathbf{F}$ and $\vec{\sigma}$ is simply a basis
for the $\mathbb{K}\left[x\right]$-module $\left\langle \left(\mathbf{F},\vec{\sigma}\right)\right\rangle $.
We again represent order bases using matrices, whose columns are the
basis elements. We only work with those order bases having minimal
or shifted minimal degrees (also referred to as a reduced order basis
in \cite{BL1997}), that is, their column degrees or shifted column
degrees are the smallest possible among all bases of the module. 
%A discussion on such minimality and its existence can be found in \cite[Chapter 2]{zhou:phd2012}.

An order basis \cite{BeLa94,BL1997} $\mathbf{P}$ of $\mathbf{F}$
with order $\vec{\sigma}$ and shift $\vec{s}$, or simply an $\left(\mathbf{F},\vec{\sigma},\vec{s}\right)$-basis,
is a basis for the module $\left\langle \left(\mathbf{F},\vec{\sigma}\right)\right\rangle $
having minimal $\vec{s}$-column degrees. If $\vec{\sigma}=\left[\sigma,\dots,\sigma\right]$
is uniform then we simply write $\left(\mathbf{F},\sigma,\vec{s}\right)$-basis.
The precise definition of an $\left(\mathbf{F},\vec{\sigma},\vec{s}\right)$-basis
is as follows. 
\begin{defn}
\label{def:orderBasis}A polynomial matrix $\mathbf{P}$ is an order
basis of $\mathbf{F}$ of order $\vec{\sigma}$ and shift $\vec{s}$,
denoted by $\left(\mathbf{F},\vec{\sigma},\vec{s}\right)$-basis,
if the following properties hold: 
\begin{enumerate}
\item $\mathbf{P}$ is a nonsingular matrix of dimension $n$ and is $\vec{s}$-column
reduced. 
\item $\mathbf{P}$ has order $\left(\mathbf{F},\vec{\sigma}\right)$ (or
equivalently, each column of $\mathbf{P}$ is in $\left\langle (\mathbf{F},\vec{\sigma})\right\rangle $). 
\item Any $\mathbf{q}\in\left\langle \left(\mathbf{F},\vec{\sigma}\right)\right\rangle $
can be expressed as a linear combination of the columns of $\mathbf{P}$,
given by $\mathbf{P}^{-1}\mathbf{q}$. 
\end{enumerate}
\end{defn}
%\begin{comment}
%Note that the module $\left\langle \left(\mathbf{F},\vec{\sigma}\right)\right\rangle $
%does not depend on the shift $\vec{s}$. 
%\end{comment}


Note that any pair of $\left(\mathbf{F},\vec{\sigma},\vec{s}\right)$-bases
$\mathbf{P}$ and $\mathbf{Q}$ are column bases of each other and
are unimodularly equivalent.

We will need to compute order bases with unbalanced shifts using Algorithm
2 from \cite{za2009}. This computation can be done efficiently as
given by the following result from \cite{za2009}.
\begin{thm}
\label{thm:unbalancedOrderBasisCost}If %the shift 
$\vec{s}$ satisfies
$\vec{s}\le0$ and $-\sum\vec{s}\le m\sigma$, then a $\left(\mathbf{F},\sigma,\vec{s}\right)$-basis
can be computed with a cost of $O^{\sim}(n^{\omega}d)$ field operations,
where $d=m\sigma/n$. 
\end{thm}

\subsection{Kernel Bases}

The kernel of $\mathbf{F}\in\mathbb{K}\left[x\right]^{m\times n}$
is the $\mathbb{F}\left[x\right]$-module 
\[
\left\{ \mathbf{p}\in\mathbb{K}\left[x\right]^{n}~|~\mathbf{F}\mathbf{p}=0\right\} 
\]
with a kernel basis of $\mathbf{F}$ being a basis of this module.
Kernel bases are closely related to order bases, as can be seen from
the following definitions. 
\begin{defn}
\label{def:kernelBasis}Given $\mathbf{F}\in\mathbb{K}\left[x\right]^{m\times n}$,
a polynomial matrix $\mathbf{N}\in\mathbb{K}\left[x\right]^{n\times*}$
is a (right) kernel basis of $\mathbf{F}$ if the following properties
hold: 
\begin{enumerate}
\item $\mathbf{N}$ is full-rank. 
\item $\mathbf{N}$ satisfies $\mathbf{F}\cdot\mathbf{N}=0$. 
\item Any $\mathbf{q}\in\mathbb{K}\left[x\right]^{n}$ satisfying $\mathbf{F}\mathbf{q}=0$
can be expressed as a linear combination of the columns of $\mathbf{N}$,
that is, there exists some polynomial vector $\mathbf{p}$ such that
$\mathbf{q}=\mathbf{N}\mathbf{p}$. 
\end{enumerate}
\end{defn}
Any pair of kernel bases $\mathbf{N}$ and $\mathbf{M}$ of $\mathbf{F}$
are column bases of each other and are unimodularly equivalent.

A $\vec{s}$-minimal kernel basis of $\mathbf{F}$ is just a kernel
basis that is $\vec{s}$-column reduced. 
\begin{defn}
Given $\mathbf{F}\in\mathbb{K}\left[x\right]^{m\times n}$, a polynomial
matrix $\mathbf{N}\in\mathbb{K}\left[x\right]^{n\times*}$ is a $\vec{s}$-minimal
(right) kernel basis of $\mathbf{F}$ if\textbf{ $\mathbf{N}$} is
a kernel basis of $\mathbf{F}$ and $\mathbf{N}$ is $\vec{s}$-column
reduced. We also call a $\vec{s}$-minimal (right) kernel basis of
$\mathbf{F}$ a $\left(\mathbf{F},\vec{s}\right)$-kernel basis. 

We will need to use the following bound on the sizes of kernel bases
from \cite{za2012}.\end{defn}
\begin{thm}
\label{thm:boundOfSumOfShiftedDegreesOfKernelBasis}Suppose $\mathbf{F}\in\mathbb{K}\left[x\right]^{m\times n}$
and $\vec{s}\in\mathbb{Z}_{\ge0}^{n}$ is a shift with entries bounding
the corresponding column degrees of $\mathbf{F}$. Then the sum of
the $\vec{s}$-column degrees of any $\vec{s}$-minimal kernel basis
of $\mathbf{F}$ is bounded by $\sum\vec{s}$.
\end{thm}
We will also need the following result from \cite{za2012} to compute
kernel bases by rows. 
\begin{thm}
\label{thm:continueComputingKernelBasisByRows}Let $\mathbf{G}=\left[\mathbf{G}_{1}^{T},\mathbf{G}_{2}^{T}\right]^{T}\in\mathbb{K}\left[x\right]^{m\times n}$
and $\vec{t}\in\mathbb{Z}^{n}$ a shift vector. If $\mathbf{N}_{1}$
is a $\left(\mathbf{G}_{1},\vec{t}\right)$-kernel basis with $\vec{t}$-column
degrees $\vec{u}$, and $\mathbf{N}_{2}$ is a $\left(\mathbf{G}_{2}\mathbf{N}_{1},\vec{u}\right)$-kernel
basis with $\vec{u}$-column degrees $\vec{v}$, then $\mathbf{N}_{1}\mathbf{N}_{2}$
is a $\left(\mathbf{G},\vec{t}\right)$-kernel basis $\vec{t}$-column
degrees $\vec{v}$. 
\end{thm}
Also recall the cost of kernel basis computation from \cite{za2012}.
\begin{thm}
\label{thm:costGeneral} Given an input matrix $\mathbf{F}\in\mathbb{K}\left[x\right]^{m\times n}$.
Let $\vec{s}=\cdeg\mathbf{F}$ and $s=\sum\vec{s}/n$ be the average
column degree of $\mathbf{F}$. Then a $\left(\mathbf{F},\vec{s}\right)$-kernel
basis can be computed with a cost of $O^{\sim}\left(n^{\omega}s\right)$
field operations.\end{thm}




\section{Column Basis via Factorization}

In this section we reduce the problem of determining a column basis
of a polynomial matrix into three separate processes. For this reduction
it turns out to be %Before discussing the efficient computation of column basis, it is 
useful to look at following relationship between column basis, kernel
basis, and unimodular matrices.
\begin{lem}
\label{lem:unimodular_kernel_columnBasis} Let $\mathbf{F}\in\mathbb{K}\left[x\right]^{m\times n}$
and suppose $\mathbf{U}\in\mathbb{K}\left[x\right]^{n\times n}$ is
a unimodular matrix such that $\mathbf{F}\mathbf{U}=\left[0,\mathbf{T}\right]$
with $\mathbf{T}$ of full column rank. Partition $\mathbf{U}=\left[\mathbf{U}_{L},\mathbf{U}_{R}\right]$
such that $\mathbf{F}\cdot\mathbf{U}_{L}=0$ and $\mathbf{F}\mathbf{U}_{R}=\mathbf{T}$.
Then 
\begin{enumerate}
\item $\mathbf{U}_{L}$ is a kernel basis of $\mathbf{F}$ and $\mathbf{T}$
is a column basis of $\mathbf{F}$. 
\item If $\mathbf{N}$ is any other kernel basis of $\mathbf{F}$, then
$\mathbf{U}^{*}=\left[\mathbf{N},~\mathbf{U}_{R}\right]$ is also
unimodular and also unimodularly transforms $\mathbf{F}$ to $\left[0,\mathbf{T}\right]$. 
\end{enumerate}
\end{lem}
\begin{proof}
Since $\mathbf{F}$ and $\left[0,\mathbf{T}\right]$ are unimodularly
equivalent with $\mathbf{F}$ having full column rank we have that
$\mathbf{T}$ is a column basis of $\mathbf{F}$. It remains to show
that $\mathbf{U}_{L}$ is a kernel basis of $\mathbf{F}$. Since $\mathbf{F}\mathbf{U}_{L}=0$,
$\mathbf{U}_{L}$ is generated by any kernel basis $\mathbf{N}$,
that is, $\mathbf{U}_{L}=\mathbf{N}\mathbf{C}$ for some polynomial
matrix $\mathbf{C}$. Let $r$ be the rank of $\mathbf{F}$, which
is also the column dimension of $\mathbf{T}$ and $\mathbf{U}_{R}$.
Then both $\mathbf{N}$ and $\mathbf{U}_{L}$ have column dimension
$n-r$. Hence $\mathbf{C}$ is a square $(n-r)\times(n-r)$ matrix.
The unimodular matrix $\mathbf{U}$ can be factored as 
\[
\mathbf{U}=\left[\mathbf{N}\mathbf{C},\mathbf{U}_{R}\right]=\left[\mathbf{N},\mathbf{U}_{R}\right]\begin{bmatrix}\mathbf{C} & 0\\
0 & I
\end{bmatrix},
\]
implying that both factors $\left[\mathbf{N},\mathbf{U}_{R}\right]$
and $\begin{bmatrix}\mathbf{C} & 0\\
0 & I
\end{bmatrix}$ are unimodular. Therefore, $\mathbf{C}$ is unimodular and $\mathbf{U}_{L}=\mathbf{N}\mathbf{C}$
is also a kernel basis. Notice that the unimodular matrix $\left[\mathbf{N},\mathbf{U}_{R}\right]$
also transforms $\mathbf{F}$ to $\left[0,\mathbf{T}\right]$. \end{proof}
\begin{rem}
\label{cor:unimodular_kernel_columnBasis2} It is interesting to see
what Lemma \ref{lem:unimodular_kernel_columnBasis} implies in the
case of unimodular matrices. Let $\mathbf{U}\in\mathbb{K}\left[x\right]^{n\times n}$
be a unimodular matrix with inverse $\mathbf{V}$, which, for a given
$k$, are partitioned as $\mathbf{U}=\left[\mathbf{U}_{L},\mathbf{U}_{R}\right]$
and $\mathbf{V}=\begin{bmatrix}\mathbf{V}_{U}\\
\mathbf{V}_{D}
\end{bmatrix}$ with $\mathbf{U}_{L}\in\mathbb{K}\left[x\right]^{n\times k}$ and
$\mathbf{V}_{U}\in\mathbb{K}\left[x\right]^{k\times n}$. Since $\mathbf{U}$
and $\mathbf{V}$ are inverses of each other we have the indentities
\begin{equation}
\mathbf{V}\mathbf{U}=\begin{bmatrix}\mathbf{V}_{U}\\
\mathbf{V}_{D}
\end{bmatrix}\begin{bmatrix}\mathbf{U}_{L},\mathbf{U}_{R}\end{bmatrix}=\begin{bmatrix}\mathbf{V}_{U}\mathbf{U}_{L} & \mathbf{V}_{U}\mathbf{U}_{R}\\
\mathbf{V}_{D}\mathbf{U}_{L} & \mathbf{V}_{D}\mathbf{U}_{R}
\end{bmatrix}=\begin{bmatrix}I & 0\\
0 & I
\end{bmatrix}.\label{inverse}
\end{equation}
Lemma \ref{lem:unimodular_kernel_columnBasis} then gives: 
\begin{enumerate}
\item $I_{r}$ is a column basis of $\mathbf{V}_{U}$ and a row basis of
$\mathbf{U}_{L}$, 
\item $I_{n-r}$ is a column basis of $\mathbf{V}_{D}$ and a row basis
of $\mathbf{U}_{R}$, 
\item $\mathbf{V}_{D}$ and $\mathbf{U}_{L}$ are kernel bases of each other, 
\item $\mathbf{V}_{U}$ and $\mathbf{U}_{R}$ are kernel bases of each other. 
\end{enumerate}
\end{rem}
\begin{lem}
\label{lem:matrixGCD} Let $\mathbf{F}\in\mathbb{K}\left[x\right]^{m\times n}$
with rank $r$. Suppose %\begin{itemize}
%\item[(i)]
 $\mathbf{N}\in\mathbb{K}\left[x\right]^{n\times(n-r)}$ is a right
kernel basis of $\mathbf{F}$ and % \item[(ii)]
$\mathbf{G}\in\mathbb{K}\left[x\right]^{r\times n}$ is a left kernel
basis of $\mathbf{N}$. %\end{itemize}
Then $\mathbf{F}=\mathbf{T}\cdot\mathbf{G}$ with $\mathbf{T}\in\mathbb{K}\left[x\right]^{m\times r}$
a column basis of $\mathbf{F}$. \end{lem}
\begin{proof}
Let $\mathbf{U}=\begin{bmatrix}\mathbf{U}_{L},\mathbf{U}_{R}\end{bmatrix}$
be a unimodular matrix with inverse $\mathbf{V}=\begin{bmatrix}\mathbf{V}_{U}\\
\mathbf{V}_{D}
\end{bmatrix}$ partitioned as in equation (\ref{inverse}) and satisfies $\mathbf{F}\cdot\mathbf{U}=\left[0,\mathbf{B}\right]$
with $\mathbf{B}\in\mathbb{K}\left[x\right]^{m\times r}$ a column
basis of $\mathbf{F}$. Then %\textbf{ 
\[
\mathbf{F}=\left[0,\mathbf{B}\right]\mathbf{U}^{-1}=\mathbf{B}\left[0,I\right]\mathbf{V}=\mathbf{B}\mathbf{V}_{D}.
\]
%}
Since $\mathbf{V}_{D}$ is a left kernel basis of\textbf{ $\mathbf{U}_{L}$},
any other left kernel basis $\mathbf{G}$ of $\mathbf{U}_{L}$ is
unimodularly equivalent to $\mathbf{V}_{D}$, that is, $\mathbf{V}_{D}=\mathbf{W}\cdot\mathbf{G}$
for some unimodular matrix $\mathbf{W}$. Thus $\mathbf{F}=\mathbf{B\cdot W}\cdot\mathbf{G}$.
Then $\mathbf{T}=\mathbf{B\cdot W}$ is a column basis of $\mathbf{F}$
since it is unimodularly equivalent to the column basis $\mathbf{B}$. 
\end{proof}
Lemma \ref{lem:matrixGCD} outlines a procedure for computing a column
basis of $\mathbf{F}$ with three main steps. The first step is to
compute a right kernel basis $\mathbf{N}$ of $\mathbf{F}$, something
which can be efficiently done using the fast kernel algorithm of \cite{za2012}.
The second step, computing a left kernel basis $\mathbf{G}$ for $\mathbf{N}$
 and the third step, computing the column basis $\mathbf{T}$ from
$\mathbf{F}$ and $\mathbf{G}$, will still require additional work
for efficient computation. Note that, while Lemma \ref{lem:matrixGCD}
does not require the bases computed to be minimal, working with minimal
kernel bases keeps the degrees well controlled, an important consideration
for efficient computation.
\begin{exmp}
Let 
\[
\mathbf{F}=\left[\begin{array}{cccc}
x^{2} & x^{2} & x+x^{2} & 1+x^{2}\\
1+x+x^{2} & x^{2} & 1+x^{2} & 1+x^{2}
\end{array}\right]~.
\]
Then the matrix 
\[
\mathbf{N}=\left[\begin{array}{cc}
x & 1\\
1 & x\\
x & 1\\
0 & x
\end{array}\right]
\]
is a right kernel basis of $\mathbf{F}$ and the matrix 
\[
\mathbf{G}=\left[\begin{array}{cccc}
1 & 0 & 1 & 0\\
\noalign{\medskip}x & {x}^{2} & 0 & 1+{x}^{2}
\end{array}\right]
\]
is a left nulllspace basis of $\mathbf{N}$. Finally the matrix 
\[
\mathbf{T}=\left[\begin{array}{cc}
x+x^{2} & 1\\
1+x^{2} & 1
\end{array}\right]
\]
satisfies $\mathbf{F}=\mathbf{T}\mathbf{G}$, and is a column basis
of $\mathbf{F}$. \end{exmp}




\section{\label{sec:computeRightFactor}Computing the Right Factor}

Let $\mathbf{N}$ be an $\left(\mathbf{F},\vec{s}\right)$-kernel
basis computed using the existing algorithm from \cite{za2012}. Consider
now the problem of computing a left $-\vec{s}$-minimal kernel basis
$\mathbf{G}$ for $\mathbf{N}$, or equivalently, a $\left(\mathbf{N}^{T},-\vec{s}\right)$-kernel
basis $\mathbf{G}^{T}$. For this problem, the fast kernel algorithm
of \cite{za2012} cannot be applied directly, since the input matrix
$\mathbf{N}^{T}$ has nonuniform row degrees and negative shift. Comparing
to the earlier problem of computing a $\vec{s}$-minimal kernel basis
$\mathbf{N}$ for $\mathbf{F}$, it is interesting to note that the
original output $\mathbf{N}$ now becomes the new input matrix $\mathbf{N}^{T}$,
while the new output matrix $\mathbf{G}$ has size bounded by $\mathbf{F}$.
In other words, the new input has degrees that match the original
output, while the new output has degrees bounded by the original input.
It is therefore reasonable to expect that the new problem can be computed
efficiently. However, we need to find some way to work with the more
complicated input degree structure. On the other hand, the simpler
output degree structure makes it easier to apply order basis computation
in order to compute a $\left(\mathbf{N}^{T},-\vec{s}\right)$-kernel
basis.


\subsection{Kernel Bases via Order Bases}

In order to see how order basis computations can be applied here,
let us first recall the following result (Lemma 3.3 \cite{za2012})
on a relationship between order bases and kernel bases.
\begin{lem}
\label{lem:orderBasisContainsNullspaceBasis}Let $\mathbf{P}=\left[\mathbf{P}_{L},\mathbf{P}_{R}\right]$
be any $\left(\mathbf{F},\sigma,\vec{s}\right)$-basis and $\mathbf{N}=\left[\mathbf{N}_{L},\mathbf{N}_{R}\right]$
be any $\vec{s}$-minimal kernel basis of $\mathbf{F}$, where $\mathbf{P}_{L}$
and $\mathbf{N}_{L}$ contain all columns from $\mathbf{P}$ and $\mathbf{N}$,
respectively, whose $\vec{s}$-column degrees are less than $\sigma$.
Then $\left[\mathbf{P}_{L},\mathbf{N}_{R}\right]$ is a $\vec{s}$-minimal
kernel basis of $\mathbf{F}$, and $\left[\mathbf{N}_{L},\mathbf{P}_{R}\right]$
is a $\left(\mathbf{F},\sigma,\vec{s}\right)$-basis.
\end{lem}
It is not difficult to extend this result to the following lemma to
accommodate our situation here.%
\begin{comment}
 For the remainder of this paper an integer vector of ones is denoted
by $\vec{e}$. 
\end{comment}

\begin{lem}
\label{lem:orderbasisContainsKernelbasisGeneralized} Given a matrix
$\mathbf{A}\in\mathbb{K}\left[x\right]^{m\times n}$ and some integer
lists $\vec{u}\in\mathbb{Z}^{n}$ and $\vec{v}\in\mathbb{Z}^{m}$
such that $\rdeg_{\vec{u}}\mathbf{A}\le\vec{v}$, or equivalently,
$\cdeg_{-\vec{v}}\mathbf{A}\le-\vec{u}$. Let $\mathbf{P}$ be a $\left(\mathbf{A},\vec{v}+1,-\vec{u}\right)$
order basis and $\mathbf{Q}$ be any $(\mathbf{A},-\vec{u})$-kernel
basis. Partition $\mathbf{P}=\left[\mathbf{P}_{L},\mathbf{P}_{R}\right]$
and $\mathbf{Q}=\left[\mathbf{Q}_{L},\mathbf{Q}_{R}\right]$ where
$\mathbf{P}_{L}$ and $\mathbf{Q}_{L}$ contain all the columns from
$\mathbf{P}$ and $\mathbf{Q}$, respectively, whose $-\vec{u}$-column
degrees are no more than $0$. Then 
\begin{itemize}
\item $\left[\mathbf{P}_{L},\mathbf{Q}_{R}\right]$ is an $(\mathbf{A},-\vec{u})$-kernel
basis, and 
\item $\left[\mathbf{Q}_{L},\mathbf{P}_{R}\right]$ is an $\left(\mathbf{A},\vec{v}+1,-\vec{u}\right)$
order basis. 
\end{itemize}
\end{lem}
\begin{proof}
We can use the same proof from Lemma 3.3 in \cite{za2012}. We know
$\cdeg_{-\vec{v}}\mathbf{A}\mathbf{P}_{L}\le\cdeg_{-\vec{u}}\mathbf{P}_{L}\le0$,
or equivalently, $\rdeg\mathbf{A}\mathbf{P}_{L}\le\vec{v}$. However
it also has order greater than $\vec{v}$ and hence $\mathbf{A}\mathbf{P}_{L}=0$.
Thus $\mathbf{P}_{L}$ is generated by the kernel basis$\mathbf{Q}_{L}$,
that is, $\mathbf{P}_{L}=\mathbf{Q}_{L}\mathbf{U}$ for some polynomial
matrix $\mathbf{U}$. On the other hand, $\mathbf{Q}_{L}$ certainly
has order $\left(\mathbf{A},\vec{v}+1\right)$ and therefore is generated
by $\mathbf{P}_{L}$, that is, $\mathbf{Q}_{L}=\mathbf{P}_{L}\mathbf{V}$
for some polynomial matrix $\mathbf{V}$. We now have $\mathbf{P}_{L}=\mathbf{P}_{L}\mathbf{V}\mathbf{U}$
and $\mathbf{Q}_{L}=\mathbf{Q}_{L}\mathbf{U}\mathbf{V}$, implying
both $\mathbf{U}$ and $\mathbf{V}$ are unimodular. The result then
follows from the unimodular equivalence of $\mathbf{P}_{L}$ and $\mathbf{Q}_{L}$
and the fact that they are $-\vec{u}$-column reduced.
\end{proof}
With the help of Lemma \ref{lem:orderbasisContainsKernelbasisGeneralized}
we can return to the problem of efficiently computing a $(\mathbf{F},\vec{s})$
kernel basis. In fact, we just need to use a special case of Lemma
\ref{lem:orderbasisContainsKernelbasisGeneralized}, where all the
elements of the kernel basis have shifted degrees bounded by $0$,
thereby making the partial kernel basis be a complete kernel basis. 
\begin{lem}
\label{lem:kernelBasisInOrderBasis} Let $\mathbf{N}$ be a $(\mathbf{F},\vec{s})$
kernel basis with $\cdeg_{\vec{s}}~\mathbf{N}=\vec{b}$. Let $\mathbf{P}=\left[\mathbf{P}_{L},\mathbf{P}_{R}\right]$
be a $\left(\mathbf{N}^{T},\vec{b}+1,-\vec{s}\right)$ order basis,
where $\mathbf{P}_{L}$ consists of all columns $\mathbf{p}$ satisfying
$\cdeg_{-\vec{s}}~\mathbf{p}\le0$. Then $\mathbf{P}_{L}$ is a $(\mathbf{N}^{T},-\vec{s})$-kernel
basis. \end{lem}
\begin{proof}
Let the rank of $\mathbf{F}$ be $r$, which is also the column dimension
of any $(\mathbf{N}^{T},-\vec{s})$-kernel basis $\mathbf{G}^{T}$.
Since both $\mathbf{F}$ and $\mathbf{G}$ are in the left kernel
of $\mathbf{N}$, we know $\mathbf{F}$ is generated by $\mathbf{G}$,
and the $-\vec{s}$-minimality of $\mathbf{G}$ ensures that the $-\vec{s}$-row
degrees of $\mathbf{G}$ are bounded by the corresponding $r$ largest
$-\vec{s}$-row degrees of $\mathbf{F}$, which are in turn bounded
by $0$ since $\cdeg\mathbf{F}\le\vec{s}$. Therefore, any $(\mathbf{N}^{T},-\vec{s})$-kernel
basis $\mathbf{G}^{T}$ satisfies $\cdeg_{-\vec{s}}\mathbf{G}^{T}\le0$.
The result now follows from Lemma \ref{lem:orderbasisContainsKernelbasisGeneralized}. 
\end{proof}
We can use Theorem \ref{thm:continueComputingKernelBasisByRows} to
compute a $\left(\mathbf{N}^{T},-\vec{s}\right)$-kernel basis by
rows. We partition $\mathbf{N}$ into $\left[\mathbf{N}_{1},\mathbf{N}_{2}\right]$
with $\vec{s}$-column degrees $\vec{b}_{1}$, $\vec{b}_{2}$ respectively.
We first compute a $\left(\mathbf{N}_{1}^{T},-\vec{s}\right)$-kernel
basis $\mathbf{Q}_{1}$ with $-\vec{s}$-column degrees $-\vec{s}_{2}$,
and then compute a $\left(\mathbf{N}_{2}^{T}\mathbf{Q}_{1},-\vec{s}_{2}\right)$-kernel
basis $\mathbf{Q}_{2}$ implying that $\mathbf{Q}_{1}\mathbf{Q}_{2}$
is a $\left(\mathbf{N}^{T},-\vec{s}\right)$-kernel basis. In order
to compute the kernel bases $\mathbf{Q}_{1}$ and $\mathbf{Q}_{2}$,
we can use order basis computation. However, we need to make sure
that the order bases we compute contain these kernel bases.
\begin{lem}
\label{lem:kernelBasisOfSubsetOfRowsContainedInOrderBasis} Let $\mathbf{N}$
be partitioned as $\left[\mathbf{N}_{1},\mathbf{N}_{2}\right]$, with
$\vec{s}$-column degrees $\vec{b}_{1}$, $\vec{b}_{2}$, respectively.
Then we have the following:
\begin{enumerate}
\item A $\left(\mathbf{N}_{1}^{T},\vec{b}_{1}+1,-\vec{s}\right)$ order
basis contains a $\left(\mathbf{N}_{1}^{T},-\vec{s}\right)$-kernel
basis whose $-\vec{s}$-column degrees are bounded by $0$. 
\item If $\mathbf{Q}_{1}$ is this $\left(\mathbf{N}_{1}^{T},-\vec{s}\right)$-kernel
basis from above and $-\vec{s}_{2}=\cdeg_{-\vec{s}}\mathbf{Q}_{1}$,
then a $\left(\mathbf{N}_{2}^{T}\mathbf{Q}_{1},\vec{b}_{2}+1,-\vec{s}_{2}\right)$-basis
contains a $\left(\mathbf{N}_{2}^{T}\mathbf{Q}_{1},-\vec{s}_{2}\right)$-kernel
basis, $\mathbf{Q}_{2}$, whose $-\vec{s}$-column degrees are bounded
by $0$. 
\item The product $\mathbf{Q}_{1}\mathbf{Q}_{2}$ is a $\left(\mathbf{N}^{T},-\vec{s}\right)$
kernel basis. 
\end{enumerate}
\end{lem}
\begin{proof}
To see that a $\left(\mathbf{N}_{1}^{T},\vec{b}_{1}+1,-\vec{s}\right)$-basis
contains a $\left(\mathbf{N}_{1}^{T},-\vec{s}\right)$-kernel basis
whose $-\vec{s}$-column degrees are bounded by 0, we just need to
show that $\cdeg_{-\vec{s}}\mathbf{\bar{Q}}_{1}\le0$ for any $\left(\mathbf{N}_{1}^{T},-\vec{s}\right)$-kernel
basis $\mathbf{\bar{Q}}_{1}$ and then apply \prettyref{lem:orderbasisContainsKernelbasisGeneralized}.
Note that there exists a polynomial matrix $\bar{\mathbf{Q}}_{2}$
such that $\mathbf{\bar{Q}}_{1}\mathbf{\bar{Q}}_{2}=\bar{\mathbf{G}}$
for any $\left(\mathbf{N}^{T},-\vec{s}\right)$-kernel basis $\bar{\mathbf{G}}$,
as $\bar{\mathbf{G}}$ satisfies $\mathbf{N}_{1}^{T}\bar{\mathbf{G}}=0$
and is therefore generated by the $\left(\mathbf{N}_{1}^{T},-\vec{s}\right)$-kernel
basis $\bar{\mathbf{Q}}_{1}$. If $\cdeg_{-\vec{s}}\mathbf{\bar{Q}}_{1}\nleq0$,
then \prettyref{lem:predictableDegree} forces 
\[
\cdeg_{-\vec{s}}\left(\bar{\mathbf{Q}}_{1}\bar{\mathbf{Q}}_{2}\right)=\cdeg_{-\vec{s}}\bar{\mathbf{G}}\nleq0,
\]
 a contradiction since we know from the proof of \prettyref{lem:kernelBasisInOrderBasis}
that $\cdeg_{-\vec{s}}\mathbf{G}^{T}\le0$. 

As before, to see that a $\left(\mathbf{N}_{2}^{T}\mathbf{Q}_{1},\vec{b}_{2}+1,-\vec{s}_{2}\right)$-basis
contains a $\left(\mathbf{N}_{2}^{T}\mathbf{Q}_{1},-\vec{s}_{2}\right)$-kernel
basis whose $-\vec{s}$-column degrees are no more than 0, we can
just show $\cdeg_{-\vec{s}_{2}}\hat{\mathbf{Q}}_{2}\le0$ for any
$\left(\mathbf{N}_{2}^{T}\mathbf{Q}_{1},-\vec{s}_{2}\right)$-kernel
basis $\hat{\mathbf{Q}}_{2}$ and then apply \prettyref{lem:orderbasisContainsKernelbasisGeneralized}.
Since $\cdeg_{\vec{s}}\mathbf{N}_{2}=\vec{b}_{2}$, we have $\rdeg_{-\vec{b}_{2}}\mathbf{N}_{2}\le-\vec{s}$
or equivalently, $\cdeg_{-\vec{b}_{2}}\mathbf{N}_{2}^{T}\le-\vec{s}.$
Then combining this with $\cdeg_{-\vec{s}}\mathbf{Q}_{1}=-\vec{s}_{2}$
we get $\cdeg_{-\vec{b}_{2}}\mathbf{N}_{2}^{T}\mathbf{Q}_{1}\le-\vec{s}_{2}$
using \prettyref{lem:predictableDegree}. Let $\hat{\mathbf{G}}=\mathbf{Q}_{1}\hat{\mathbf{Q}}_{2}$,
which is a $\left(\mathbf{N}^{T},-\vec{s}\right)$-kernel basis by
\prettyref{thm:continueComputingKernelBasisByRows}. Note that $\cdeg_{-\vec{s}_{2}}\hat{\mathbf{Q}}_{2}=\cdeg_{-\vec{s}}\mathbf{Q}_{1}\hat{\mathbf{Q}}_{2}=\cdeg_{-\vec{s}}\hat{\mathbf{G}}\le0.$
\end{proof}

\subsection{\label{sub:kernelBasisViaOrderBasisByRows}Efficient Computation
of Kernel Bases}

Now that we can correctly compute a $\left(\mathbf{N}^{T},-\vec{s}\right)$-kernel
basis by rows with the help of order basis computation using \prettyref{lem:kernelBasisOfSubsetOfRowsContainedInOrderBasis},
we need to look at how to do it efficiently. One major difficulty
is that the order $\vec{b}+1$, or equivalently, the $\vec{s}$-row
degrees of $\mathbf{N}_{1}^{T}$ are nonuniform and can have degree
as large as $\sum\vec{s}$. To overcome this, we separate the rows
of $\mathbf{N}^{T}$ into blocks according to their $\vec{s}$-row
degrees, and then work with these blocks one by one successively using
\prettyref{thm:continueComputingKernelBasisByRows}. 

\input{AlgorithmNullspaceBasisReverse.tex}

Let $k$ be the column dimension of $\mathbf{N}$ and $\xi$ be an
upper bound of $\sum\vec{s}$. Since 
\[
\sum\cdeg_{\vec{s}}\mathbf{N}=\sum\vec{b}\le\sum\vec{s}\le\xi
\]
 by \prettyref{thm:boundOfSumOfShiftedDegreesOfKernelBasis}, at most
$\frac{k}{c}$ columns of $\mathbf{N}$ have $\vec{s}$-column degrees
greater than or equal to $\frac{c~\xi}{k}$ for any $c\ge1$. Without
loss of generality we can assume that the rows of $\mathbf{N}^{T}$
are arranged in decreasing $\vec{s}$-row degrees. We divide $\mathbf{N}^{T}$
into $\log k$ row blocks according to the $\vec{s}$-row degrees
of its rows, or equivalently, divide $\mathbf{N}$ into blocks of
columns according to the $\vec{s}$-column degrees. Let 
\[
\mathbf{N}=\left[\mathbf{N}_{1},\mathbf{N}_{2},\cdots,\mathbf{N}_{\log k-1},\mathbf{N}_{\log k}\right]
\]
with $\mathbf{N}_{\log k},\mathbf{N}_{\log k-1},\dots,\mathbf{N}_{2},\mathbf{N}_{1}$
having $\vec{s}$-column degrees in the range $\left[0,2\xi/k\right]$,
$(2\xi/k,4\xi/k],$ $(4\xi/k,8\xi/k],\ ...,$ $(\xi/4,\xi/2],$ $(\xi/2,\xi].$
Let $\vec{\sigma}_{i}=\left[\xi/2^{i-1}+1,\dots,\xi/2^{i-1}+1\right]$
with the same dimension as the row dimension of $\mathbf{N}_{i}$
and 
\[
\vec{\sigma}=\left[\vec{\sigma}_{\log k},\vec{\sigma}_{\log k-1},\dots,\vec{\sigma}_{1}\right]
\]
 be the orders in the order basis computation.

To further simply our task, we also make the order of our problem
in each block uniform. Rather than of using $\mathbf{N}^{T}$ as the
input matrix, we instead use 
\begin{eqnarray*}
\hat{\mathbf{N}} & =\begin{bmatrix}\hat{\mathbf{N}}_{1}\\
\vdots\\
\hat{\mathbf{N}}_{\log k}
\end{bmatrix}= & x^{\vec{\sigma}-\vec{b}-1}\begin{bmatrix}\mathbf{N}_{1}^{T}\\
\vdots\\
\mathbf{N}_{\log k}^{T}
\end{bmatrix}=x^{\vec{\sigma}-\vec{b}-1}\mathbf{N}^{T}
\end{eqnarray*}
so that a $\left(\hat{\mathbf{N}},\vec{\sigma},-\vec{s}\right)$ order
basis is a $\left(\mathbf{N}^{T},\vec{b}+1,-\vec{s}\right)$ order
basis.

In order to compute a $\left(\mathbf{N}^{T},-\vec{s}\right)$-kernel
basis we determine a series of kernel bases via a series of order
basis computations as follows:
\begin{enumerate}
\item Let $\vec{s}_{1}=\vec{s}$. Compute an $\left(\hat{\mathbf{N}}_{1},\vec{\sigma}_{1},-\vec{s}_{1}\right)$
order basis $\mathbf{P}_{1}$ using Algorithm 2 from \cite{za2009}
for order basis computation with unbalanced shift. Partitioned $\mathbf{P}_{1}$
as $\mathbf{P}_{1}=\left[\mathbf{G}_{1},\mathbf{Q}_{1}\right]$, where
$\mathbf{G}_{1}$ is a $\left(\hat{\mathbf{N}}_{1},-\vec{s}_{1}\right)$-kernel
basis by \prettyref{lem:kernelBasisOfSubsetOfRowsContainedInOrderBasis}.
Set $\tilde{\mathbf{G}}_{1}=\mathbf{G}_{1}$ and $\vec{s}_{2}=-\cdeg_{-\vec{s}}\mathbf{G}_{1}$. 
\item Compute an $\left(\hat{\mathbf{N}}_{2}\tilde{\mathbf{G}}_{1},\vec{\sigma}_{2},-\vec{s}_{2}\right)$
order basis $\mathbf{P}_{2}$ and partition $\mathbf{P}_{2}=\left[\mathbf{G}_{2},\mathbf{Q}_{2}\right]$
with $\mathbf{G}_{2}$ a $\left(\hat{\mathbf{N}}_{2},-\vec{s}_{2}\right)$
kernel basis. Set $\vec{s}_{3}=-\cdeg_{-\vec{s}_{2}}\mathbf{G}_{2}$
and $\tilde{\mathbf{G}}_{2}=\tilde{\mathbf{G}}_{1}\mathbf{G}_{2}$.
\item Continuing this process, at each step $i$ we compute a $\left(\hat{\mathbf{N}}_{i}\tilde{\mathbf{G}}_{i-1},\vec{\sigma}_{i},-\vec{s}_{i}\right)$
order basis $\mathbf{P}_{i}$ and then partition $\mathbf{P}_{i}=\left[\mathbf{G}_{i},\mathbf{Q}_{i}\right]$
with $\mathbf{G}_{i}$ a $\left(\hat{\mathbf{N}}_{i}\tilde{\mathbf{G}}_{i-1},-\vec{s}_{i}\right)$
kernel basis. Let $\tilde{\mathbf{G}}_{i}=\prod_{j=1}^{i}\mathbf{G}_{i}=\tilde{\mathbf{G}}_{i-1}\mathbf{G}_{i}$. 
\item Return $\tilde{\mathbf{G}}_{\log k}$, a $\left(\mathbf{N}^{T},-\vec{s}\right)$-kernel
basis. 
\end{enumerate}
This process of computing a $\left(\mathbf{N}^{T},-\vec{s}\right)$-kernel
basis is formally given in Algorithm \ref{alg:minimalNullspaceBasisReverse}.


\subsection{Complexity of Left Kernel Computation}

The cost of Algorithm \ref{alg:minimalNullspaceBasisReverse} is dominated
by the order basis computations and the multiplications $\hat{\mathbf{N}}_{i}\tilde{\mathbf{G}}_{i-1}$
and $\tilde{\mathbf{G}}_{i-1}\mathbf{G}_{i}$. Let $s=\xi/n$.
\begin{lem}
An $\left(\hat{\mathbf{N}}_{i}\tilde{\mathbf{G}}_{i-1},\vec{\sigma}_{i},-\vec{s}_{i}\right)$
order basis can be computed with a cost of $O^{\sim}\left(n^{\omega}s\right)$. \end{lem}
\begin{proof}
Note that $\mathbf{N}_{i}$ has less than $2^{i}$ columns. Otherwise,
\[
\sum\cdeg_{\vec{s}}\mathbf{N}_{i}>2^{i}\xi/2^{i}=\xi,
\]
contradicting with 
\[
\sum\cdeg_{\vec{s}}\mathbf{N}=\sum\vec{b}\le\sum\vec{s}\le\xi.
\]
It follows that $\hat{\mathbf{N}}_{i}$, and therefore $\hat{\mathbf{N}}_{i}\tilde{\mathbf{G}}_{i-1}$,
also have less than $2^{i}$ rows. We also have $\vec{\sigma}_{i}=\left[\xi/2^{i-1}+1,\dots,\xi/2^{i-1}+1\right]$
with entries in $\Theta\left(\xi/2^{i}\right)$. Therefore, Algorithm
2 from \cite{za2009} for order basis computation with unbalanced
shift can be used with a cost of $O^{\sim}\left(n^{\omega}s\right)$. \end{proof}
\begin{lem}
The multiplications $\hat{\mathbf{N}}_{i}\tilde{\mathbf{G}}_{i-1}$
can be done with a cost of $O^{\sim}\left(n^{\omega}s\right)$.\end{lem}
\begin{proof}
The dimension of $\hat{\mathbf{N}}_{i}$ is bounded by $2^{i-1}\times n$
and $\sum\rdeg_{\vec{s}}\hat{\mathbf{N}}_{i}\le2^{i-1}\cdot\xi/2^{i-1}=\xi$.
We also have $\cdeg_{-\vec{s}}\tilde{\mathbf{G}}_{i-1}\le0$, or equivalently,
$\rdeg\tilde{\mathbf{G}}_{i-1}\le\vec{s}$. We can now use Theorem
\ref{thm:multiplyUnbalancedMatrices} to multiply $\tilde{\mathbf{G}}_{i-1}^{T}$
and $\hat{\mathbf{N}}_{i}^{T}$ with a cost of $O^{\sim}\left(n^{\omega}s\right)$.\end{proof}
\begin{lem}
The multiplication $\tilde{\mathbf{G}}_{i-1}\mathbf{G}_{i}$ can be
done with a cost of $O^{\sim}\left(n^{\omega}s\right)$. \end{lem}
\begin{proof}
We know $\cdeg_{-\vec{s}}\tilde{\mathbf{G}}_{i-1}=-\vec{s}_{i}$,
and $\cdeg_{-\vec{s}_{i}}\mathbf{G}_{i}=-\vec{s}_{i+1}\le0.$ In other
words, $\rdeg\mathbf{G}_{i}\le\vec{s}_{i}$, and $\rdeg_{\vec{s}_{i}}\tilde{\mathbf{G}}_{i-1}\le\vec{s}$,
hence we can again use Theorem \ref{thm:multiplyUnbalancedMatrices}
to multiply $\mathbf{G}_{i}^{T}$ and $\tilde{\mathbf{G}}_{i-1}^{T}$
with a cost of $O^{\sim}\left(n^{\omega}s\right)$. \end{proof}
\begin{lem}
\label{lem:costKernelBasisReverse}Given an input matrix $\mathbf{M}\in\mathbb{K}\left[x\right]^{k\times n}$,
a shift $\vec{s}\in\mathbb{Z}^{n}$, and an upper bound $\xi\in\mathbb{Z}$
such that 
\begin{itemize}
\item $\sum\rdeg_{\vec{s}}\mathbf{M}\le\xi$,
\item $\sum\vec{s}\le\xi$,
\item and any $\left(\mathbf{M},-\vec{s}\right)$-kernel basis having row
degrees bounded by $\vec{s}$, or equivalently, having $-\vec{s}$-column
degrees bounded by 0.
\end{itemize}
Then \prettyref{alg:minimalNullspaceBasisReverse} costs $O^{\sim}\left(n^{\omega}s\right)$
field operations to compute a $\left(\mathbf{M},-\vec{s}\right)$-kernel
basis.

Note that $\xi$ can be simply set to $\sum\vec{s}$.\end{lem}
\begin{thm}
A right factor $\mathbf{G}$ satisfying $\mathbf{F}=\mathbf{TG}$
for a column basis $\mathbf{T}$ can be computed with a cost of $O^{\sim}\left(n^{\omega}s\right)$. \end{thm}




\section{Computing a Column Basis}

Once a right factor $\mathbf{G}$ of $\mathbf{F}$ has been computed,
we are in a position to determine a column basis $\mathbf{T}$ using
the equation $\mathbf{F}=\mathbf{T}\mathbf{G}$. In order to do so
efficiently, however, the degree of $\mathbf{T}$ cannot be too large.
We see that this is the case from the following lemma. 
\begin{lem}
\label{lem:colBasisdegreeBoundByRdegOfRightFactor} Let $\mathbf{F}$
and $\mathbf{G}$ be as before and $\vec{t}=-\rdeg_{-\vec{s}}\mathbf{G}$.
Then 
\begin{itemize}
\item the column degrees of $\mathbf{T}$ are bounded by the corresponding
entries of $\vec{t}$; 
\item if $\vec{t}$ has $r$ entries and $\vec{s}^{~\prime}$ is the list
of the $r$ largest entries of $\vec{s}$, then $\vec{t}\le\vec{s}^{~\prime}$. 
\end{itemize}
\end{lem}
\begin{proof}
Since $\mathbf{G}$ is $-\vec{s}$-row reduced, and $\rdeg_{-\vec{s}}\mathbf{F}\le0$,
by Lemma \ref{lem:predictableDegree} $\rdeg_{-\vec{t}}\mathbf{T}\le0$,
or equivalently, $\mathbf{T}$ has column degrees bounded by $\vec{t}$.

Let $\mathbf{G}^{\prime}$ be the $-\vec{s}$-row Popov form of $\mathbf{G}$
and the square matrix $\mathbf{G}^{\prime\prime}$ consist of only
the columns of $\mathbf{G}^{\prime}$ that contains pivot entries,
and has the rows permuted so the pivots are in the diagonal. Let $\vec{s}^{~\prime\prime}$
be the list of the entries in $\vec{s}$ that correspond to the columns
of $\mathbf{G}^{\prime\prime}$ in $\mathbf{G}^{\prime}$. Note that
$\rdeg_{-\vec{s}^{~\prime\prime}}\mathbf{G}^{\prime\prime}=-\vec{t}^{~\prime\prime}$
is just a permutation of $-\vec{t}$ with the same entries. By the
definition of shifted row degree, $-\vec{t}^{~\prime\prime}$ is the
sum of $-\vec{s}^{~\prime\prime}$ and the list of the diagonal pivot
degrees, which are nonnegative. Therefore, $-\vec{t}^{~\prime\prime}\ge-\vec{s}^{~\prime\prime}$.
The result then follows as $\vec{t}$ is a permutation of $\vec{t}^{~\prime\prime}$
and $\vec{s}^{\ \prime}$ consists of the largest entries of $\vec{s}$. 
\end{proof}
Having a bound on the column degrees of $\mathbf{T}$ determined,
we are now ready to compute $\mathbf{T}$. This is done again by computing
a kernel basis using an order basis computation as before. 
\begin{lem}
Let $\vec{t}^{*}=\left[0,\dots,0,\vec{t}\right]\in\mathbb{Z}^{m+r}$.
Then any $\left(\left[\mathbf{F}^{T},\mathbf{G}^{T}\right],-\vec{t}^{*}\right)$-kernel
basis has the form $\begin{bmatrix}V\\
\bar{\mathbf{T}}
\end{bmatrix}$, where $V\in\mathbb{K}^{m\times m}$ is a unimodular matrix and $\left(\bar{\mathbf{T}}V^{-1}\right)^{T}$
is a column basis of $\mathbf{F}$. \end{lem}
\begin{proof}
Note first that the matrix $\begin{bmatrix}-I\\
\mathbf{T}^{T}
\end{bmatrix}$ is a kernel basis of $\left[\mathbf{F}^{T},\mathbf{G}^{T}\right]$
and is therefore unimodularly equivalent to any other kernel basis.
Hence any other kernel basis has the form $\begin{bmatrix}-I\\
\mathbf{T}^{T}
\end{bmatrix}U=\begin{bmatrix}V\\
\bar{\mathbf{T}}
\end{bmatrix}$, with $U$ and $V=-U$ unimodular. Thus $\mathbf{T}=\left(\bar{\mathbf{T}}V^{-1}\right)^{T}$.
Also note that the $-\vec{t}^{*}$ minimality forces the unimodular
matrix $V$ in any $\left(\left[\mathbf{F}^{T},\mathbf{G}^{T}\right],-\vec{t}^{*}\right)$-kernel
basis to be degree 0, the same degree as $I$. \end{proof}
\begin{exmp}
Let 
\[
\mathbf{F}=\left[\begin{array}{cccc}
x^{2} & x^{2} & x+x^{2} & 1+x^{2}\\
1+x+x^{2} & x^{2} & 1+x^{2} & 1+x^{2}
\end{array}\right]
\]
with 
\[
\mathbf{G}=\left[\begin{array}{cccc}
1 & 0 & 1 & 0\\
\noalign{\medskip}x & {x}^{2} & 0 & 1+{x}^{2}
\end{array}\right]
\]
being a minimal left kernel basis of a right kernel basis of $\mathbf{F}$.
In order to compute the column basis $\mathbf{T}$ satisfying $\mathbf{F}=\mathbf{T}\mathbf{G}$,
first we can determine $\cdeg\mathbf{T}\le\vec{t}=\left[2,0\right]$
from Lemma \ref{lem:colBasisdegreeBoundByRdegOfRightFactor}. Then
we can compute a $\left[0,0,-\vec{t}\right]$-minimal left kernel
basis of $\begin{bmatrix}\mathbf{F}\\
\mathbf{G}
\end{bmatrix}$. The matrix 
\[
\left[V,\bar{\mathbf{T}}\right]=\left[\begin{array}{cccc}
\noalign{\medskip}1 & 0 & x+{x}^{2} & 1\\
1 & 1 & 1+x & 0
\end{array}\right]
\]
is such a left kernel basis. A column basis can then be computed as
by 
\[
\mathbf{T}=V^{-1}\bar{\mathbf{T}}=\left[\begin{array}{cc}
x+x^{2} & 1\\
1+{x}^{2} & 1
\end{array}\right].
\]

\end{exmp}
In order to compute a $\left(\left[\mathbf{F}^{T},\mathbf{G}^{T}\right],-\vec{t}^{*}\right)$-kernel
basis, we can again use order basis computation as before, as we again
have an order basis that contains a $\left(\left[\mathbf{F}^{T},\mathbf{G}^{T}\right],-\vec{t}^{*}\right)$-kernel
basis.
\begin{lem}
Any $\left(\left[\mathbf{F}^{T},\mathbf{G}^{T}\right],\vec{s}+\vec{e},-\vec{t}^{*}\right)$
order basis contains a $\left(\left[\mathbf{F}^{T},\mathbf{G}^{T}\right],-\vec{t}^{*}\right)$-kernel
basis whose $-\vec{t}^{*}$-row degrees are bounded by 0. \end{lem}
\begin{proof}
As before, Lemma \ref{lem:orderbasisContainsKernelbasisGeneralized}
can be used here. We just need to show that a $\left(\left[\mathbf{F}^{T},\mathbf{G}^{T}\right],-\vec{t}^{*}\right)$-kernel
basis has $-\vec{t}^{*}$-row degrees no more than $0$. This follows
since $\rdeg_{-\vec{t}^{*}}\begin{bmatrix}I\\
\mathbf{T}^{T}
\end{bmatrix}\le0$. 
\end{proof}
In order to compute a $\left(\left[\mathbf{F}^{T},\mathbf{G}^{T}\right],-\vec{t}^{*}\right)$
kernel basis efficiently, we notice that we have the same type of
problem as in Section \ref{sub:kernelBasisViaOrderBasisByRows} and
hence we can again use Algorithm \ref{alg:minimalNullspaceBasisReverse}. 
\begin{lem}
\label{lem:costOfKernelBasisReversedForLeftFactor}A $\left(\left[\mathbf{F}^{T},\mathbf{G}^{T}\right],-\vec{t}^{*}\right)$-kernel
basis can be computed using \prettyref{alg:minimalNullspaceBasisReverse}
with a cost of $O^{\sim}\left(n^{\omega}s\right)$, where $s=\xi/n$
is the average column degree of $\mathbf{F}$ as before. \end{lem}
\begin{proof}
Just use the algorithm with input $\left(\left[\mathbf{F}^{T},\mathbf{G}^{T}\right],\vec{t}^{*},\xi\right)$.
We can verify the conditions on the input are satisfied.
\begin{itemize}
\item To see that $\sum\rdeg_{\vec{t}^{*}}\left[\mathbf{F}^{T},\mathbf{G}^{T}\right]\le\xi$,
note that from $\vec{t}=-\rdeg_{-\vec{s}}\mathbf{G}$ and \prettyref{lem:columnDegreesRowDegreesSymmetry}
$\cdeg_{\vec{t}}\mathbf{G}\le\vec{s}$, or equivalently, $\rdeg_{\vec{t}}\mathbf{G}^{T}\le\vec{s}$.
Since we also have $\rdeg\mathbf{F}^{T}\le\vec{s}$, it follows that
$\rdeg_{\vec{t}^{*}}\left[\mathbf{F}^{T},\mathbf{G}^{T}\right]\le\vec{s}$. 
\item The second condition $\sum\vec{t}^{*}\le\xi$ follows from \prettyref{lem:colBasisdegreeBoundByRdegOfRightFactor}.
\item The third condition holds since $\begin{bmatrix}-I\\
\mathbf{T}^{T}
\end{bmatrix}$ is a kernel basis with row degrees bounded by $\vec{t}^{*}$.
\end{itemize}
\end{proof}
With a $\left(\left[\mathbf{F}^{T},\mathbf{G}^{T}\right],-\vec{t}^{*}\right)$-kernel
basis $\begin{bmatrix}V\\
\bar{\mathbf{T}}
\end{bmatrix}$ computed, a column basis is then given by $\mathbf{T}~=~\left(\bar{\mathbf{T}}V^{-1}\right)^{T}$.

The complete algorithm for computing a column basis is then given
in Algorithm \ref{alg:colBasis}.

\input{AlgorithmColumnBasis.tex} 
\begin{thm}
\label{thm:columnBasisCost1}A column basis $\mathbf{T}$ of $\mathbf{F}$
can be computed with a cost of $O^{\sim}\left(n^{\omega}s\right)$,
where $s=\xi/n$ is the average column degree of $\mathbf{F}$ as
before. \end{thm}
\begin{proof}
The cost is dominated by the cost of the three kernel basis computations
in the algorithm. The first one is handled by the algorithm from \cite{za2012}
and \prettyref{thm:costGeneral}, while the remaining two are handled
by \prettyref{alg:minimalNullspaceBasisReverse}, \prettyref{lem:costKernelBasisReverse}
and \prettyref{lem:costOfKernelBasisReversedForLeftFactor}.\end{proof}




\section{\label{sec:successiveColBasisComputation}A Simple Improvement}

When the input matrix $\mathbf{F}$ has column dimension much larger
$n$ than the row dimension $m$, we can separate $\mathbf{F}=\left[\mathbf{F}_{1},\mathbf{F}_{2},\dots,\mathbf{F}_{n/m}\right]$
to $n/m$ blocks, each with dimension $m\times m$, assuming without
loss of generality $n$ is a multiple of $m$, and the columns are
arranged in increasing degrees. We then do a series of column basis
computations. First we compute a column basis $\mathbf{T}_{1}$ of
$\left[\mathbf{F}_{1},\mathbf{F}_{2}\right]$. Then compute a column
basis $\mathbf{T}_{2}$ of $\left[\mathbf{T}_{1},\mathbf{F}_{3}\right]$.
Repeating this process, at step $i$, we compute a column basis $\mathbf{T}_{i}$
of $\left[\mathbf{T}_{i-1},\mathbf{F}_{i+1}\right]$, until $i=n/m-1$,
when a column basis of $\mathbf{F}$ is computed. 
\begin{lem}
Let $\bar{s}_{i}=\left(\sum\cdeg\mathbf{F}_{i}\right)/m$. Then at
step $i$, computing a column basis $\mathbf{T}_{i}$ of $\left[\mathbf{T}_{i-1},\mathbf{F}_{i+1}\right]$
can be done with a cost of $O^{\sim}\left(m^{\omega}(\bar{s}_{i}+\bar{s}_{i+1})/2\right)$
field operations.\end{lem}
\begin{proof}
From Lemma \ref{lem:colBasisdegreeBoundByRdegOfRightFactor}, the
column basis $\mathbf{T}_{i-1}$ of $\left[\mathbf{F}_{1},\dots,\mathbf{F}_{i}\right]$
has column degrees bounded by the largest column degrees of $\mathbf{F}_{i}$,
hence $\sum\cdeg\mathbf{T}_{i-1}\le\sum\cdeg\mathbf{F}_{i}$. The
lemma then follows by combining this with the result from Theorem
\ref{thm:columnBasisCost1} that a column basis $\mathbf{T}_{i}$
of $\left[\mathbf{T}_{i-1},\mathbf{F}_{i+1}\right]$ can be computed
with a cost of $O^{\sim}\left(m^{\omega}\hat{s}_{i}\right)$, where
\[
\hat{s}_{i}=\left(\sum\cdeg\mathbf{T}_{i-1}+\sum\cdeg\mathbf{F}_{i+1}\right)/2m\le\left(\bar{s}_{i}+\bar{s}_{i+1}\right)/2.
\]
 \end{proof}
\begin{thm}
If $s=\left(\sum\cdeg\mathbf{F}\right)/n$, then \label{thm:finalCollBasisCost}A
column basis of $\mathbf{F}$ can be computed with a cost of $O^{\sim}\left(m^{\omega}s\right)$. \end{thm}
\begin{proof}
Summing up the cost of all the column basis computations, 
\begin{eqnarray*}
 &  & \sum_{i=1}^{n/m-1}O^{\sim}\left(m^{\omega}\left(\bar{s}_{i}+\bar{s}_{i+1}\right)/2\right)\\
 & \subset & O^{\sim}\left(m^{\omega}\left(\sum_{i=1}^{n/m}\bar{s}_{i}\right)\right)=O^{\sim}\left(nm^{\omega-1}s\right),
\end{eqnarray*}
 since $\sum\cdeg\mathbf{F}=\sum_{i=1}^{n/m}\left(m\bar{s}_{i}\right)=ns.$\end{proof}
\begin{rem}
In this section, the computational efficiency is improved by reducing
the original problem to about $n/m$ subproblems whose column dimensions
are close to the row dimension $m$. This is done by successive column
basis computations. Note that we can also reduce the column dimension
by using successive order basis computations, and only do a column
basis computation at the very last step. The computational complexity
of using order basis computation to reduce the column dimension would
remain the same, but in practice it may be more efficient since order
basis computations are simpler. \end{rem}






\section{Conclusion}

In this paper we have given a fast, deterministic algorithm for the
computation of a column basis for $\mathbf{F}$ having complexity
$O^{\sim}\left(n^{\omega}s\right)$ field operations in $\mathbb{K}$
with $s$ an upper bound for the average average column degree of
$\mathbf{F}$. Our methods rely on a special factorization of $\mathbf{F}$
into a left kernel and a right column basis. These in turn are computed
via fast kernel and fast order basis algorithm of \cite{za2012,ZL2012}.
When these computations involve the multiplication of polynomial matrices
with unbalanced degrees then they use the fast algorithms for unbalanced
multiplication given in \cite{za2012}.

For a number of applications of column basis computation, including
efficient deterministic computations of column reduced form, Popov
normal form, Hermite normal form, and determinant, we refer the readers
to the thesis \cite{zhou:phd2012}.

%In a later publication we show how our methods can be extended for fast methods for finding 



%\bibliographystyle{plainnat}
\bibliographystyle{plain}
\bibliography{paper}

\end{document}
